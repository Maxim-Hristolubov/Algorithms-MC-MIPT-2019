\documentclass[a4paper,12pt]{article} % добавить leqno в [] для нумерации слева

%%% Работа с русским языком
\usepackage{cmap}					% поиск в PDF
\usepackage{mathtext} 				% русские буквы в формулах
\usepackage[T2A]{fontenc}			% кодировка
\usepackage[utf8]{inputenc}			% кодировка исходного текста
\usepackage[english,russian]{babel}	% локализация и переносы

%%% Дополнительная работа с математикой
\usepackage{amsmath,amsfonts,amssymb,amsthm,mathtools} % AMS
\usepackage{icomma} % "Умная" запятая: $0,2$ --- число, $0, 2$ --- перечисление

%% Номера формул
\mathtoolsset{showonlyrefs=true} % Показывать номера только у тех формул, на которые есть \eqref{} в тексте.

%% Раисунки (автоматы)
\usepackage{tikz}

%% Шрифты
\usepackage{euscript}	 % Шрифт Евклид
\usepackage{mathrsfs} % Красивый матшрифт
\usepackage{ upgreek }

%% Свои команды
\DeclareMathOperator{\sgn}{\mathop{sgn}}
\usepackage{ dsfont }

%% Перенос знаков в формулах (по Львовскому)
\newcommand*{\hm}[1]{#1\nobreak\discretionary{}
	{\hbox{$\mathsurround=0pt #1$}}{}}

%%% Заголовок
\author{771 группа, Христолюбов Максим}
\title{Домашнее задание по АМВ}
\date{\today}

\begin{document} % конец преамбулы, начало документа
	
	\maketitle
	
\section{Задача 1}
\hspace{5mm}
$d=e^{-1}(\mod(p-1)(q-1))=3^{-1}(\mod{352})=235(\mod352)$

$y = x^{e}(\mod N)=41^3=352\cdot 41(\mod391)=105(\mod391)$

$x = y^d(\mod N)=105^{235}(\mod391)=41(\mod391)$

\section{Задача 2}
\hspace{5mm}
Злоумышленник может умножить $d$ на $e$ и получить число, которое, как известно, равно $1$ по модулю $(p-1)(q-1)$. Вычислив число $de-1$ он может перебирать все делители этого числа, которых полиномиальное от $\log{(de-1)}O(\log(de))$ количество и которые являются кандидатами на $(p-1)(q-1)$. Для каждого делителя он может перебрать все его разложение на $2$ множителя, которые являются кандидатами на $p-1$ и $q-1$, которых тоже полиномиальное от $\log(p-1)(q-1)=O(\log(de))$ количество. Кандидаты на $p$ и $q$ оцениваются по длине, и если их суммарная длина не больше длины $N$, то их произведение может быть равно $N$, поэтому они перемножаются, и среди находятся те, что $pq=N$. Перемножение чисел, чья запись не больше $N$, $O(\log N)$. Таким образом за полиномиальное время $O(\log^2(de)\log N)$ находится разложение $N$ на $p$ и $q$.

\section{Задача 3}
\hspace{5mm}
То есть $A^{ed}=A(\mod{N})=A^{2021d}=A(\mod 25)=A^d=A(\mod 25)$, $d=\phi(25)=20$.

\section{Задача 4}
\hspace{5mm}
a) Можно сравнить между собой два средних элемента массива. Если $a_k\leq a_{k+1}$, то значит, $a_k$ точно не максимум. Тогда сравним $a_{k+3}$ и $a_{k+4}$ и в дальнейшем будем двигаться вправо от центра (как далее поясняется). В противоположном случае будем сравнивать $a_{k-3}$ и $a_{k-2}$, и двигаться влево от центра, аналогично тому как двигались бы вправо.

Если оказалось, что $a_{k+3}\geq a_{k+4}$, то следующими $2$ шагами можно найти горку. Сравним $a_{k+1}$ и $a_{k+2}$, если $a_{k+1}\geq a_{K+2}$, то $a_{k+1}$ горка, иначе сравним $a_{k+2}$ и $a_{k+3}$, и зависимости от результата горкой окажется либо $a_{k+2}$, либо $a_{k+3}$. 

Если оказалось, что $a_{k+3}< a_{k+4}$? то сравним $a_{k+6}$ и $a_{k+7}$, и проделаем с ними те же действия, что и в предыдущем абзаце. Либо $a_{k+6}\geq a_{k+7}$, то следующими $2$ шагами находим горку, иначе продолжаем со сравнением $a_{k+9}$ и $a_{k+10}$ и т.~д.

Так будет продолжаться либо пока горка не найдется, либо не дойдем до правого конца массива. Тогда сравнив $a_{n-1}$ и $a_n$, и либо $a_n$ окажется горкой, либо $a_{n-1}$ окажется горкой, поскольку на предыдущем шаге горка не нашлась, а значит, либо $a_{n-2}\leq a_{n-1}$, либо, если они не сравнивались, то $2$ сравнениями $a_{n-2}$ c $a_{n-3}$, которое больше $a_{n-4}$, и с $a_{n-1}$, горка будет найдена. Всего на движение от центра до края уйдет не более $\frac{n+2}{3}$ сравнений и еще $2$ в худшем случае. Значит, асимптотика $\Omega(\frac{n+8}{3})$.

\section{Задача 5}
\hspace{5mm}
Заменим граф $G$ на $\overline{G}$, в котором все ребра из $G$ остутствуют, а отсутсвующие присутствуют. Тогда разбиение графа $G$ на $2$ клики будет совпадать с разбиением графа $\overline{G}$ на $2$ доли. Проверить можно ли разбить так граф займет полинмиальное время: будем перебирать вершины и рассматривать соседние вершины, относя их в другую долю. Если так разделить получилось, то $G$ тоже можно разбить на $2$ клики, а если нет, то значит где то в $\overline{G}$ есть цикл из вершин, соединенные попарно, нечетной длины, поэтому и разбить на $2$ доли его нельзя, а значит и $G$ не рахбить на $2$ клики. Так как $P\neq NP$, то эта задача не $NPC$.

\section{Задача 6}
\hspace{5mm}
Условие на граф эквивалентно тому, что существуют пути из $s$ в $t$ длины $10$ и $11$, так как, все остальные пути можно получить перемещаясь вперед-назад по какому-то ребру. Проверить есть ли такие пути можно за полиномиальное время. Будем строить всевозможные пути длины $10$ и $11$ из $s$ и смотреть есть ли среди них те, которые оканчиваются в $t$. Построить один путь можно за константое время переходами по спискам смежности. Всего путей длины $11$ не более $n^11$, где $n$~---~кол-во вершин графа. Значит, их все можно перебрать за полиномиальное время и определить есть ли пути такой длины. Значит задача лежит в $P$, аследовательно и в $co-NP$.



\section{Задача 7}
\hspace{5mm}
Отсортируем ребра по убыванию веса за $O(m\log m)$. Так как сложность раскраски это максимальный вес ребра, чьи вершины закрашены в один цвет, то единственное о чем нужно заботиться~---~так это о том, чтоб максимальное количество ребер с максимальным весом были между вершинами разных цветов. Возьмем первое ребро из списка, отсортированного по убыванию, и раскрасим его вершины в $2$ разных цвета. Так же поступим со следующим. Так будем продолжать, пока не окажется, что обе вершины ребра уже окрашены в один цвет. Это будет значить, что нельзя раскрасить граф лучше, так как любое изменение в уже проведенной раскраске приведет к тому, что ребро еще большего веса окажется окрашено в один цвет, и сложность раскраски будет больше, чем с ребром, которое оказалось раскрашено в один цвет на данном этапе. Остальные вершины можно раскрасить как угодно, на сложность раскраски это никак не повлияет. Таким образом действий было совершено $O(m\log m)+O(m)=O(m\log m)$ как и требовалось.


$a=\{a_1,\ldots, a_n\}, b =\{b_1,\ldots, b_n\}$

$DTW(a,b) =\ ?$

Построим матрицу $\Omega=\Omega(a,b)$: $\Omega_{ij}=|a_i-b_j|$.

Путь из $(1,1)$ в $(n,n)$ в матрице $\Omega$:

$\pi = \{\pi_r\}=\{(i_r,j_r)\}$, причем путь непрерывен и монотонен.

Cтоимость пути $\pi$: $S(\Omega,\pi)=\sum\limits_{(i,j)\in\pi}\Omega_{ij}$

$DTW(a,b)=\underset{\pi}{\min}\ S(\Omega(a,b),\pi)$


1) Для каждого класса построим матрицу $W$ попарных расстояний между объектами этого класса. и понизим ее размерность методом главных компонент.

Центроидом класса $D_e$ по расстоянию $\rho$ приближенно считаем

$z_e=\underset{z\in D_e}{argmin}\sum\limits_{s_i\in D_e}{\rho(W^T s_i,W^T z)}$

2) Построим матрицу $X=D\times D_e$, в которой у каждого объекта признаками выступают расстояния $\rho(x,z_e)$.

3) Будем решать задачу классификации на полученных признаках $X$ методом $k$ ближайших соседей.


По матрице $\Omega$ построим матрицу $S$ кратчайших стоимостей \\ путей из $(1,1)$ в $(i,j)$ в $\Omega$, таким образом, что

$S_{11}=\Omega_{11}$

$S_{ij}=\Omega_{ij}+\min(S_{i(j-1)},S_{(i-1)j},S_{(i-1)(j-1)})$

Тогда $DTW(a,b)=S_{nn}$




\end{document} % конец документа
