\documentclass[a4paper,12pt]{article} % добавить leqno в [] для нумерации слева

%%% Работа с русским языком
\usepackage{cmap}					% поиск в PDF
\usepackage{mathtext} 				% русские буквы в формулах
\usepackage[T2A]{fontenc}			% кодировка
\usepackage[utf8]{inputenc}			% кодировка исходного текста
\usepackage[english,russian]{babel}	% локализация и переносы

%%% Дополнительная работа с математикой
\usepackage{amsmath,amsfonts,amssymb,amsthm,mathtools} % AMS
\usepackage{icomma} % "Умная" запятая: $0,2$ --- число, $0, 2$ --- перечисление

%% Номера формул
\mathtoolsset{showonlyrefs=true} % Показывать номера только у тех формул, на которые есть \eqref{} в тексте.

%% Раисунки (автоматы)
\usepackage{tikz}

%% Шрифты
\usepackage{euscript}	 % Шрифт Евклид
\usepackage{mathrsfs} % Красивый матшрифт
\usepackage{ upgreek }

%% Свои команды
\DeclareMathOperator{\sgn}{\mathop{sgn}}
\usepackage{ dsfont }

%% Перенос знаков в формулах (по Львовскому)
\newcommand*{\hm}[1]{#1\nobreak\discretionary{}
	{\hbox{$\mathsurround=0pt #1$}}{}}

%%% Заголовок
\author{771 группа, Христолюбов Максим}
\title{Домашнее задание по АМВ}
\date{\today}

\begin{document} % конец преамбулы, начало документа
	
	\maketitle
	
\section{Задача 1}
\hspace{5mm}
Будем использовать несколько последовательных поисков в ширину. Так как он занимает $O(n+m)$ времени, то весь алгоритм $O(n+m)$.

Найдем поиском в ширину расстояние от $a$ до $v_1$, а потом до $v_2$, а так же от $v_1$ и $v_2$ до $b$. После этого поиском в ширину найдем расстояние от $a$ до $b$. Если $S(a,b)=S(a,v_1)+S(v_1,b)+1$ или $S(a,b)=S(a,v_2)+S(v_2,b)+1$, то ребро $v_1v_2$ принадлежит одному из кратчайших путей от $a$ до $b$, иначе~---~не принадлежит.



\section{Задача 2}
\hspace{5mm}
Можно либо предварительно стянуть все ребра, которые имеют вес $0$, то есть вместо $2$ вершин, соединенных ребром веса $0$, поставить одну вершину, которая будет соединяться со всеми вершинами, с которыми были соединены изначальные $2$ вершины. Тогда в графе не будет ребер $0$ длины и можно будет запустить $BFS$.

Или можно каждый раз перед тем как находить вершины, находящиеся на расстоянии $k+1$, рассматривая соседей вершин, которые находятся на расстоянии $k$, запускать поиск в глубину из каждой вершины, которая на расстоянии $k$, по ребрам $0$ длины и добавлять найденные вершины ко множеству вершин, находящихся на расстоянии $k$.

\section{Задача 3}
\hspace{5mm}

\section{Задача 4}
\subsection{a)}
\hspace{5mm}
Алгоритм Флойда-Уоршелла итак корректно работает в графах с отрицательными ребрами без отрицательных циклов. Если же в графе есть отрицательный цикл, то для какого-то $i$ $d_{ii}^(n)<0$, но в этих графах задача о поиске кратчайшего пути не имеет смысла.

\subsection{б)}
\hspace{5mm}
$D^{(0)}=\begin{pmatrix}
	0 & 1 & 3 & -1 \\
	\infty & 0 & \infty & -2 \\
	\infty & -2 & 0 & -4 \\
	\infty & \infty & \infty & 0 \\
\end{pmatrix}$

$D^{(1)}=\begin{pmatrix}
0 & 1 & 3 & -1 \\
\infty & 0 & \infty & -2 \\
\infty & -2 & 0 & -4 \\
\infty & \infty & \infty & 0 \\
\end{pmatrix}$, $D^{(2)}=\begin{pmatrix}
0 & 1 & 3 & -1 \\
\infty & 0 & \infty & -2 \\
\infty & -2 & 0 & -4 \\
\infty & \infty & \infty & 0 \\
\end{pmatrix}$

$D^{(3)}=\begin{pmatrix}
0 & 1 & 3 & -1 \\
\infty & 0 & \infty & -2 \\
\infty & -2 & 0 & -4 \\
\infty & \infty & \infty & 0 \\
\end{pmatrix}$, $D^{(4)}=\begin{pmatrix}
0 & 1 & 3 & -1 \\
\infty & 0 & \infty & -2 \\
\infty & -2 & 0 & -4 \\
\infty & \infty & \infty & 0 \\
\end{pmatrix}$

\section{Задача 5}
\subsection{b)}
\hspace{5mm}
Метки перестанут обновляться после $k$-ой фазы алгоритма, где $k$~---~кол-во ребер в кратчайшем пути с наибольшим кол-вом ребер в нем, так как до этой фазы этот кратчайший путь еще не был найден, а после этой фазы все кратчайшие пути длины меньше-равных $k$ уже найдены, то есть найдены все кратчайшие пути и метки обновляться не будут. Перебором и внимательным вглядыванием кол-во ребер в кратчайшем пути до $D$~---~$0$, $C$~---~$1$, $E$~---~$2$, $B$~---~$2$, $H$~---~$2$, $G$~---~$3$, $F$~---~$4$, $A$~---~$3$, значит, после $4$ фазы метки обновляться не будут.

\subsection{c)}
\hspace{5mm}
Получается, что наибольшая реберная длина кратчайшего пути до некоторой точки в графе равна $|V|-1$. Так как веса всех ребер равны $1$, то реберная длина равняется длине кратчайшего пути. Так как всего вершин $|V|$, то все они должны входить в этот самый длинный кратчайший путь между $2$ точками, причем в этом пути, конечно, нет самопересечений, следовательно,граф представляет из себя цепочку из $|V|$ вершин с $|V|-1$ ребрами между ними.

\section{Задача 6}
\hspace{5mm}
Их количество будет равно количеству вершин. Так как из определения частичного порядка следует, что $aRb, bRa$ следует, что  $a=b$, а если $a$ и $b$ достижимы друг из друга, то это значит, что существует такая последовательность вершин $с_1,c_2\ldots c_k$, что есть двусторонние ребра $(a,c_1),(c_1,c_2)\ldots (c_{k-1},c_k),(c_k,b)$, значит, так как на графе определено отношение порядка из условия, то $aRb$ и $bRa$, следовательно, $a=b$. Отсюда следует, что все вершины из одной компоненты сильной связности совпадают, то есть в каждой компоненты сильной связности только одна вершина.

\section{Задача 7}
\subsection{a)}
\hspace{5mm}
Пусть при поиске в глубину вход в вершину $u$ произошел раньше, тогда прежде чем выйти из $u$ поиск в глубину дойдет до вершины $v$ и выйдет из нее, и время выхода из $u$ будет больше чем из $v$. 

Пусть вход в вершину $v$ произошел раньше, тогда так как $u$ не достижима из $v$, то поиск в глубину выйдет из вершины $v$, так и не войдя в вершину $u$. Позже поиск в глубину дойдет до вершины $u$ и потом выйдет из нее, и время выхода из $u$ будет больше чем из $v$.

\subsection{b)}
\hspace{5mm}
Если бы ребро графа шло бы от вершины $v$ с меньшим номером к $u$ с большим номером, тогда это значило бы, что существует такой поиск в глубину, что время выхода из $u$ больше, чем время выхода из $v$, но $u$ достижимо из $v$, и $v$ не достижимо из $u$, так как граф ациклический, значит, как было доказано ранее время выхода из $v$ больше, чем время выхода из $u$ для любого поиска в глубину~---~противоречие. Значит, такого быть не может и ребра идут от вершин с большим номером к меньшим.

\section{Задача 8}
\subsection{a)}
\hspace{5mm}
При инвертации графа достижимость $u$ из $v$ заменяется на достижимость $v$ из $u$. Если $u\sim v$, то они оба достижимы из друг друга, и при инвертации они останутся быть достижимыми друг из друга, то есть $u'\sim v'$. Значит, все вершины, находящиеся в компонентах сильной связности останутся в них же, компоненты сильной связности не изменяться.

\subsection{b)}
\hspace{5mm}
Для инвертации графа потребуется $O(m)$, для поиска в глубину в инвертированном графе потребуется $O(n+m)$, для поиска в глубину в изначальном графе еще $O(n+m)$. Всего $O(n+m)$. 

Для алгоритма требуется инвертировать граф, но можно не создавать новый граф, а просто инверсно интерпретировать элементы матрицы смежности. При поиске в глубину требуется хранить вершины в стеке, в худшем случае в нем будут все $n$ вершин. Так же требуется запоминать время выхода для каждой вершины, еще $O(n)$ памяти. Отсортировать вершины можно используя константную дополнительную память в дополнение к памяти занимаемой самим сортируемым множеством. Значит всего уйдет памяти $O(n)$.

\subsection{c)}
\hspace{5mm}
Да, так как в конденсации графа нет циклов (является деревом, в общем случае лесом), то всегда можно отсортировать вершины, чтобы ребра шли только от вершин с большим номером к меньшим.




\end{document} % конец документа