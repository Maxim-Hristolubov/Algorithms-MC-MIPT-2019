\documentclass[a4paper,12pt]{article} % добавить leqno в [] для нумерации слева

%%% Работа с русским языком
\usepackage{cmap}					% поиск в PDF
\usepackage{mathtext} 				% русские буквы в формулах
\usepackage[T2A]{fontenc}			% кодировка
\usepackage[utf8]{inputenc}			% кодировка исходного текста
\usepackage[english,russian]{babel}	% локализация и переносы

%%% Дополнительная работа с математикой
\usepackage{amsmath,amsfonts,amssymb,amsthm,mathtools} % AMS
\usepackage{icomma} % "Умная" запятая: $0,2$ --- число, $0, 2$ --- перечисление

%% Номера формул
\mathtoolsset{showonlyrefs=true} % Показывать номера только у тех формул, на которые есть \eqref{} в тексте.

%% Раисунки (автоматы)
\usepackage{tikz}

%% Шрифты
\usepackage{euscript}	 % Шрифт Евклид
\usepackage{mathrsfs} % Красивый матшрифт
\usepackage{ upgreek }

%% Свои команды
\DeclareMathOperator{\sgn}{\mathop{sgn}}
\usepackage{ dsfont }

%% Перенос знаков в формулах (по Львовскому)
\newcommand*{\hm}[1]{#1\nobreak\discretionary{}
	{\hbox{$\mathsurround=0pt #1$}}{}}

%%% Заголовок
\author{771 группа, Христолюбов Максим}
\title{Домашнее задание по АМВ}
\date{\today}

\begin{document} % конец преамбулы, начало документа
	
	\maketitle
	
\section{Задача 1}
\hspace{5mm}
(i) $f=2+2+3=7$

(ii)

\begin{center}
	\begin{tikzpicture}[scale=0.2]
	\tikzstyle{every node}+=[inner sep=0pt]
	\draw [black] (4.4,-32.1) circle (3);
	\draw (4.4,-32.1) node {$s$};
	\draw [black] (18,-44.4) circle (3);
	\draw [black] (18,-32.1) circle (3);
	\draw [black] (18,-19.6) circle (3);
	\draw [black] (34.6,-32.1) circle (3);
	\draw [black] (34.6,-44.4) circle (3);
	\draw [black] (33.9,-19.6) circle (3);
	\draw [black] (51.2,-19.6) circle (3);
	\draw [black] (51.2,-32.1) circle (3);
	\draw (51.2,-32.1) node {$t$};
	\draw [black] (51.2,-44.4) circle (3);
	\draw [black] (7.33,-31.462) arc (98.86377:81.13623:25.119);
	\fill [black] (15.07,-31.46) -- (14.36,-30.84) -- (14.2,-31.83);
	\draw (11.2,-30.66) node [above] {$3$};
	\draw [black] (6.988,-33.615) arc (57.37679:38.37009:37.749);
	\fill [black] (16.23,-41.98) -- (16.13,-41.04) -- (15.34,-41.66);
	\draw (12.97,-36.92) node [above] {$5$};
	\draw [black] (17.261,-29.196) arc (-170.12528:-189.87472:19.508);
	\fill [black] (17.26,-22.5) -- (16.63,-23.21) -- (17.62,-23.38);
	\draw (16.47,-25.85) node [left] {$1$};
	\draw [black] (20.962,-31.626) arc (97.10565:82.89435:43.156);
	\fill [black] (31.64,-31.63) -- (30.91,-31.03) -- (30.78,-32.02);
	\draw (26.3,-30.79) node [above] {$2$};
	\draw [black] (20.941,-43.813) arc (98.82625:81.17375:34.926);
	\fill [black] (31.66,-43.81) -- (30.95,-43.2) -- (30.79,-44.18);
	\draw (26.3,-42.9) node [above] {$2$};
	\draw [black] (6.031,-29.584) arc (144.29057:120.88276:31.215);
	\fill [black] (15.36,-21.01) -- (14.41,-21) -- (14.93,-21.85);
	\draw (9.24,-24.33) node [above] {$3$};
	\draw [black] (17.006,-41.575) arc (-166.60868:-193.39132:14.357);
	\fill [black] (17.01,-34.92) -- (16.33,-35.59) -- (17.31,-35.82);
	\draw (16.12,-38.25) node [left] {$4$};
	\draw [black] (20.968,-19.17) arc (96.34174:83.65826:45.099);
	\fill [black] (30.93,-19.17) -- (30.19,-18.58) -- (30.08,-19.58);
	\draw (25.95,-18.39) node [above] {$2$};
	\draw [black] (36.783,-18.776) arc (102.68399:77.31601:26.265);
	\fill [black] (48.32,-18.78) -- (47.65,-18.11) -- (47.43,-19.09);
	\draw (42.55,-17.63) node [above] {$1$};
	\draw [black] (52.392,-22.345) arc (16.50354:-16.50354:12.338);
	\fill [black] (52.39,-29.35) -- (53.1,-28.73) -- (52.14,-28.45);
	\draw (53.4,-25.85) node [right] {$3$};
	\draw [black] (36.581,-29.848) arc (136.70888:117.25149:44.103);
	\fill [black] (48.49,-20.88) -- (47.55,-20.8) -- (48.01,-21.69);
	\draw (41.15,-24.36) node [above] {$2$};
	\draw [black] (49.929,-29.392) arc (-162.25443:-197.74557:11.62);
	\fill [black] (49.93,-22.31) -- (49.21,-22.92) -- (50.16,-23.22);
	\draw (48.88,-25.85) node [left] {$3$};
	\draw [black] (48.235,-20.056) arc (-83.07623:-96.92377:47.163);
	\fill [black] (36.86,-20.06) -- (37.6,-20.65) -- (37.72,-19.66);
	\draw (42.55,-20.9) node [below] {$2$};
	\draw [black] (30.951,-20.148) arc (-81.89006:-98.10994:35.453);
	\fill [black] (20.95,-20.15) -- (21.67,-20.76) -- (21.81,-19.77);
	\draw (25.95,-21) node [below] {$1$};
	\draw [black] (18.867,-22.468) arc (11.6757:-11.6757:16.713);
	\fill [black] (18.87,-29.23) -- (19.52,-28.55) -- (18.54,-28.35);
	\draw (19.71,-25.85) node [right] {$3$};
	\draw [black] (16.17,-21.976) arc (-39.49225:-55.33443:45.572);
	\fill [black] (6.92,-30.48) -- (7.86,-30.43) -- (7.3,-29.61);
	\draw (12.86,-27.03) node [below] {$2$};
	\draw [black] (15.051,-32.646) arc (-82.44271:-97.55729:29.285);
	\fill [black] (7.35,-32.65) -- (8.08,-33.25) -- (8.21,-32.26);
	\draw (11.2,-33.4) node [below] {$2$};
	\draw [black] (15.301,-43.093) arc (-118.96011:-145.29301:27.615);
	\fill [black] (5.97,-34.65) -- (6.02,-35.6) -- (6.84,-35.03);
	\draw (9.14,-39.9) node [below] {$3$};
	\draw [black] (31.671,-45.045) arc (-80.28313:-99.71687:31.824);
	\fill [black] (20.93,-45.04) -- (21.63,-45.67) -- (21.8,-44.69);
	\draw (26.3,-46) node [below] {$3$};
	\draw [black] (37.441,-43.442) arc (104.64922:75.35078:21.588);
	\fill [black] (48.36,-43.44) -- (47.71,-42.76) -- (47.46,-43.72);
	\draw (42.9,-42.24) node [above] {$4$};
	\draw [black] (48.271,-45.045) arc (-80.28313:-99.71687:31.824);
	\fill [black] (37.53,-45.04) -- (38.23,-45.67) -- (38.4,-44.69);
	\draw (42.9,-46) node [below] {$1$};
	\draw [black] (50.058,-41.633) arc (-164.40609:-195.59391:12.586);
	\fill [black] (50.06,-34.87) -- (49.36,-35.5) -- (50.32,-35.77);
	\draw (49.1,-38.25) node [left] {$2$};
	\draw [black] (51.999,-34.988) arc (10.61073:-10.61073:17.715);
	\fill [black] (52,-41.51) -- (52.64,-40.82) -- (51.65,-40.63);
	\draw (52.8,-38.25) node [right] {$1$};
	\draw [black] (49.138,-34.279) arc (-45.10856:-61.81722:50.77);
	\fill [black] (37.28,-43.06) -- (38.23,-43.12) -- (37.75,-42.24);
	\draw (44.54,-39.6) node [below] {$1$};
	\draw [black] (36.659,-42.219) arc (134.94008:118.13414:50.482);
	\fill [black] (48.51,-33.44) -- (47.57,-33.37) -- (48.04,-34.25);
	\draw (41.26,-36.89) node [above] {$1$};
	\draw [black] (33.458,-41.633) arc (-164.40609:-195.59391:12.586);
	\fill [black] (33.46,-34.87) -- (32.76,-35.5) -- (33.72,-35.77);
	\draw (32.5,-38.25) node [left] {$2$};
	\draw [black] (35.402,-34.987) arc (10.64963:-10.64963:17.656);
	\fill [black] (35.4,-41.51) -- (36.04,-40.82) -- (35.06,-40.63);
	\draw (36.21,-38.25) node [right] {$1$};
	\draw [black] (31.66,-32.693) arc (-81.07528:-98.92472:34.552);
	\fill [black] (20.94,-32.69) -- (21.65,-33.31) -- (21.81,-32.32);
	\draw (26.3,-33.61) node [below] {$1$};
	\draw [black] (37.485,-31.285) arc (102.37119:77.62881:25.273);
	\fill [black] (37.49,-31.29) -- (38.37,-31.6) -- (38.16,-30.63);
	\draw (42.9,-30.2) node [above] {$2$};
	\draw [black] (37.6,-32.1) -- (48.2,-32.1);
	\fill [black] (48.2,-32.1) -- (47.4,-31.6) -- (47.4,-32.6);
	\draw (42.9,-32.6) node [below] {$1$};
	\draw [black] (49.141,-21.781) arc (-44.97815:-61.06148:53.133);
	\fill [black] (37.26,-30.72) -- (38.21,-30.77) -- (37.72,-29.9);
	\draw (44.52,-27.17) node [below] {$1$};
	\draw [black] (33.4,-29.357) arc (-162.50975:-191.07982:13.997);
	\fill [black] (33.01,-22.46) -- (32.37,-23.15) -- (33.35,-23.34);
	\draw (32.19,-25.96) node [left] {$1$};
	\draw [black] (34.874,-22.434) arc (14.13809:-7.72766:17.8);
	\fill [black] (35.25,-29.18) -- (35.85,-28.45) -- (34.86,-28.32);
	\draw (35.97,-25.76) node [right] {$2$};
	\draw [black] (19.845,-29.735) arc (139.65945:116.68683:36.161);
	\fill [black] (31.17,-20.83) -- (30.23,-20.75) -- (30.68,-21.64);
	\draw (24.05,-24.22) node [above] {$1$};
	\draw [black] (31.54,-21.45) -- (20.36,-30.25);
	\fill [black] (20.36,-30.25) -- (21.3,-30.14) -- (20.68,-29.36);
	\draw (26.96,-26.35) node [below] {$2$};
	\end{tikzpicture}
\end{center}

(iii)

Нет, так как в остаточном графе есть путь из $s$ в $t$, да и можно получить поток мощности $9$, пустив потоки мощности $3$ по верхней, средней и нижней ветке.

(v)

Нужно запустить алгоритм Форда-Фалкерсона и посмотреть на все достижимые из начала вершины в последнем остаточном дереве. Эти вершины будут в левой части минимального разреза, остальные в правой. Действительно, если вершины не достижимы из начала, то к ним нельзя пустить дополнительный поток, а значит ребра в этом разрезе~---~узкое место для максимального потока в графе, в котором сумма ребер минимальна и ограничивает максимальный поток.

\section{Задача 3}
\hspace{5mm}
(i)

Можно задать веса всех ребер $1$ и найти алгоритмом Эдмондса — Карпа минимальный поток для каждой пары вершин, определить узкие место для этих потоков. Это претенденты на минимальный разрез в графе, из них можно выбрать минимальный. Тогда мощность соответствующего потока будет минимальным количеством ребер, которые нужно удалить, чтобы граф стал несвязным. По полученному число уже можно проверить является ли граф реберно $k$-связным. Полиномиальное количество нахождений минимального разреза, на которые уходит полиномиальное время является полиномиальным, то есть алгоритм полиномиален.

(ii)

Заменим все вершины на пары вершин, соединенных односторонним ребром веса, равным $1$. В одну вершину будут входить все ребра, которые входили в изначальную вершину, а из второй исходить все ребра, выходящие из изначальной. Веса всех ребер сделаем равными бесконечности. Тогда проверка на реберную $k$-связность в новом графе будет эквивалентна проверке на вершинную $k$-связность изначального графа, поскольку топология нового графа не изменилась. Действительно, если в новом графе удалить все ребра соответствующие изначальным вершинам, которые нужно было удалить для обеспечения несвязности изначального графа, то новый станет не связным. В обратную сторону это тоже верно. Таким образом задача сводится к реберной $k$-связности.

\section{Задача 4}
\hspace{5mm}

\begin{center}
	\begin{tikzpicture}[scale=0.2]
	\tikzstyle{every node}+=[inner sep=0pt]
	\draw [black] (29.9,-18.6) circle (3);
	\draw (29.9,-18.6) node {$s$};
	\draw [black] (11.3,-29.7) circle (3);
	\draw (11.3,-29.7) node {$1$};
	\draw [black] (23.4,-29.7) circle (3);
	\draw (23.4,-29.7) node {$2$};
	\draw [black] (29.9,-38.9) circle (3);
	\draw (29.9,-38.9) node {$t$};
	\draw [black] (27.32,-20.14) -- (13.88,-28.16);
	\fill [black] (13.88,-28.16) -- (14.82,-28.18) -- (14.31,-27.32);
	\draw (19.6,-23.65) node [above] {$n$};
	\draw [black] (14.3,-29.7) -- (20.4,-29.7);
	\fill [black] (20.4,-29.7) -- (19.6,-29.2) -- (19.6,-30.2);
	\draw (17.35,-29.2) node [above] {$1$};
	\draw [black] (25.13,-32.15) -- (28.17,-36.45);
	\fill [black] (28.17,-36.45) -- (28.12,-35.51) -- (27.3,-36.08);
	\draw (27.24,-32.93) node [right] {$n$};
	\draw [black] (13.99,-31.03) -- (27.21,-37.57);
	\fill [black] (27.21,-37.57) -- (26.72,-36.77) -- (26.27,-37.66);
	\draw (19.61,-34.8) node [below] {$n$};
	\draw [black] (28.38,-21.19) -- (24.92,-27.11);
	\fill [black] (24.92,-27.11) -- (25.75,-26.67) -- (24.89,-26.17);
	\draw (26,-22.91) node [left] {$n$};
	\draw [black] (29.9,-21.6) -- (29.9,-35.9);
	\fill [black] (29.9,-35.9) -- (30.4,-35.1) -- (29.4,-35.1);
	\draw (30.4,-28.75) node [right] {$3n$};
	\end{tikzpicture}
\end{center}

Если алгоритм Форда-Фалкерсона в этом графе на первом шаге построит путь $s\rightarrow1\rightarrow2\rightarrow t$, на втором шаге $s\rightarrow2\rightarrow1\rightarrow t$ и т.~д., то есть на всех нечетных шагах $s\rightarrow1\rightarrow2\rightarrow t$, ан а четных $s\rightarrow2\rightarrow1\rightarrow t$, тогда прежде чем вершины $1$ и $2$ станут не достижимы из $s$ потребуется не менее $2n$ шагов. Если рассматривать такую последовательность графов $n=1,2...$, то размер записи ребер и вершин --- константа $C$, а размер записи весов ребер $\log n$, всего $l=\log n +С$. Значит $2n=2\cdot2^{-C}\cdot 2^l= C_1 2^l$ экспоненциально от длины записи графа. Значит, алгоритм не полиномиален.

\section{Задача 5}
\hspace{5mm}
Заменим все вершины на пары вершин, соединенных односторонним ребром веса, равным весу изначальной вершины. В одну вершину будут входить все ребра, которые входили в изначальную вершину, а из второй исходить все ребра, выходящие из изначальной. Веса всех ребер сделаем равными бесконечности. После на новом графе стандартным алгоритмом поиска максимального потока можно найти максимальный поток изначального графа. Топология сети не изменилась, поэтому максимальный поток в новом графе будет проходить через те же ребра и ребра-бывшие-вершины, что и в изначальном графе.

\section{Задача 6}
\hspace{5mm}
Для всех таких двудольных графов теорема Холла выполняется, поскольку, если бы мощность $A:A\subset L$ была бы больше, чем мощность $N(A)$ (их соединяет $|A|\cdot d$ ребер), тогда бы получалось, что у какой-то вершины из $N(A)$ больше $d$ смежных ребер. Значит, полное паросочетание существует. Теперь для раскраски нужно найти это полное паросочетание. Найдем первую пару алгоритмом с семинара и выбросим из графа эту пару вершин за время $O(VE^2)$. Так проделаем $d$ раз и найдем искомую раскраску. Сложность алгоритма $O(dVE^2)=O(V^2 E^2)$.


\section{Задача 7}
\hspace{5mm}
(i)

\begin{center}
	\begin{tikzpicture}[scale=0.2]
	\tikzstyle{every node}+=[inner sep=0pt]
	\draw [black] (24.5,-18.2) circle (3);
	\draw (24.5,-18.2) node {$-1$};
	\draw [black] (11.3,-29.7) circle (3);
	\draw (11.3,-29.7) node {$s$};
	\draw [black] (39.9,-18.2) circle (3);
	\draw (39.9,-18.2) node {$4$};
	\draw [black] (24.5,-39.6) circle (3);
	\draw (24.5,-39.6) node {$-4$};
	\draw [black] (39.9,-39.6) circle (3);
	\draw (39.9,-39.6) node {$3$};
	\draw [black] (51,-29.7) circle (3);
	\draw (51,-29.7) node {$t$};
	\draw [black] (27.5,-18.2) -- (36.9,-18.2);
	\fill [black] (36.9,-18.2) -- (36.1,-17.7) -- (36.1,-18.7);
	\draw [black] (42.14,-37.6) -- (48.76,-31.7);
	\fill [black] (48.76,-31.7) -- (47.83,-31.86) -- (48.5,-32.6);
	\draw (46.46,-35.14) node [below] {$3$};
	\draw [black] (38.55,-20.879) arc (-27.89464:-43.58492:74.871);
	\fill [black] (38.55,-20.88) -- (37.73,-21.35) -- (38.62,-21.82);
	\draw [black] (24.5,-36.6) -- (24.5,-21.2);
	\fill [black] (24.5,-21.2) -- (24,-22) -- (25,-22);
	\draw [black] (39.9,-21.2) -- (39.9,-36.6);
	\fill [black] (39.9,-36.6) -- (40.4,-35.8) -- (39.4,-35.8);
	\draw [black] (37.273,-38.153) arc (-121.95795:-166.56249:27.601);
	\fill [black] (37.27,-38.15) -- (36.86,-37.31) -- (36.33,-38.15);
	\draw [black] (13.56,-27.73) -- (22.24,-20.17);
	\fill [black] (22.24,-20.17) -- (21.31,-20.32) -- (21.96,-21.07);
	\draw (18.91,-24.44) node [below] {$1$};
	\draw [black] (13.7,-31.5) -- (22.1,-37.8);
	\fill [black] (22.1,-37.8) -- (21.76,-36.92) -- (21.16,-37.72);
	\draw (16.9,-35.15) node [below] {$4$};
	\draw [black] (41.98,-20.36) -- (48.92,-27.54);
	\fill [black] (48.92,-27.54) -- (48.72,-26.62) -- (48,-27.31);
	\draw (44.92,-25.42) node [left] {$4$};
	\end{tikzpicture}
\end{center}

\begin{center}
	\begin{tikzpicture}[scale=0.2]
	\tikzstyle{every node}+=[inner sep=0pt]
	\draw [black] (24.5,-18.2) circle (3);
	\draw (24.5,-18.2) node {$-1$};
	\draw [black] (11.3,-29.7) circle (3);
	\draw (11.3,-29.7) node {$s$};
	\draw [black] (39.9,-18.2) circle (3);
	\draw (39.9,-18.2) node {$4$};
	\draw [black] (24.5,-39.6) circle (3);
	\draw (24.5,-39.6) node {$-4$};
	\draw [black] (39.9,-39.6) circle (3);
	\draw (39.9,-39.6) node {$3$};
	\draw [black] (51,-29.7) circle (3);
	\draw (51,-29.7) node {$t$};
	\draw [black] (27.5,-18.2) -- (36.9,-18.2);
	\fill [black] (36.9,-18.2) -- (36.1,-17.7) -- (36.1,-18.7);
	\draw [black] (42.14,-37.6) -- (48.76,-31.7);
	\fill [black] (48.76,-31.7) -- (47.83,-31.86) -- (48.5,-32.6);
	\draw (46.46,-35.14) node [below] {$3$};
	\draw [black] (38.55,-20.879) arc (-27.89464:-43.58492:74.871);
	\fill [black] (38.55,-20.88) -- (37.73,-21.35) -- (38.62,-21.82);
	\draw [black] (24.5,-36.6) -- (24.5,-21.2);
	\fill [black] (24.5,-21.2) -- (24,-22) -- (25,-22);
	\draw [black] (25.337,-36.72) arc (161.34754:127.1729:35.245);
	\fill [black] (25.34,-36.72) -- (26.07,-36.12) -- (25.12,-35.8);
	\draw (29.54,-26.03) node [left] {$4$};
	\draw [black] (22.1,-37.8) -- (13.7,-31.5);
	\fill [black] (13.7,-31.5) -- (14.04,-32.38) -- (14.64,-31.58);
	\draw (18.9,-34.15) node [above] {$4$};
	\draw [black] (48.92,-27.54) -- (41.98,-20.36);
	\fill [black] (41.98,-20.36) -- (42.18,-21.28) -- (42.9,-20.59);
	\draw (45.98,-22.48) node [right] {$4$};
	\draw [black] (39.9,-21.2) -- (39.9,-36.6);
	\fill [black] (39.9,-36.6) -- (40.4,-35.8) -- (39.4,-35.8);
	\draw [black] (37.273,-38.153) arc (-121.95795:-166.56249:27.601);
	\fill [black] (37.27,-38.15) -- (36.86,-37.31) -- (36.33,-38.15);
	\draw [black] (13.56,-27.73) -- (22.24,-20.17);
	\fill [black] (22.24,-20.17) -- (21.31,-20.32) -- (21.96,-21.07);
	\draw (18.91,-24.44) node [below] {$1$};
	\end{tikzpicture}
\end{center}


\begin{center}
	\begin{tikzpicture}[scale=0.2]
	\tikzstyle{every node}+=[inner sep=0pt]
	\draw [black] (24.5,-18.2) circle (3);
	\draw (24.5,-18.2) node {$-1$};
	\draw [black] (11.3,-29.7) circle (3);
	\draw (11.3,-29.7) node {$s$};
	\draw [black] (39.9,-18.2) circle (3);
	\draw (39.9,-18.2) node {$4$};
	\draw [black] (24.5,-39.6) circle (3);
	\draw (24.5,-39.6) node {$-4$};
	\draw [black] (39.9,-39.6) circle (3);
	\draw (39.9,-39.6) node {$3$};
	\draw [black] (51,-29.7) circle (3);
	\draw (51,-29.7) node {$t$};
	\draw [black] (27.5,-18.2) -- (36.9,-18.2);
	\fill [black] (36.9,-18.2) -- (36.1,-17.7) -- (36.1,-18.7);
	\draw [black] (49.468,-32.276) arc (-35.03687:-61.5041:20.001);
	\fill [black] (49.47,-32.28) -- (48.6,-32.64) -- (49.42,-33.22);
	\draw (47.42,-36.21) node [below] {$2$};
	\draw [black] (38.55,-20.879) arc (-27.89464:-43.58492:74.871);
	\fill [black] (38.55,-20.88) -- (37.73,-21.35) -- (38.62,-21.82);
	\draw [black] (24.5,-36.6) -- (24.5,-21.2);
	\fill [black] (24.5,-21.2) -- (24,-22) -- (25,-22);
	\draw [black] (25.337,-36.72) arc (161.34754:127.1729:35.245);
	\fill [black] (25.34,-36.72) -- (26.07,-36.12) -- (25.12,-35.8);
	\draw (29.54,-26.03) node [left] {$4$};
	\draw [black] (22.1,-37.8) -- (13.7,-31.5);
	\fill [black] (13.7,-31.5) -- (14.04,-32.38) -- (14.64,-31.58);
	\draw (18.9,-34.15) node [above] {$4$};
	\draw [black] (48.92,-27.54) -- (41.98,-20.36);
	\fill [black] (41.98,-20.36) -- (42.18,-21.28) -- (42.9,-20.59);
	\draw (45.98,-22.48) node [right] {$4$};
	\draw [black] (41.613,-37.139) arc (141.82722:121.6318:25.783);
	\fill [black] (41.61,-37.14) -- (42.5,-36.82) -- (41.71,-36.2);
	\draw (43.71,-33.34) node [above] {$1$};
	\draw [black] (39.9,-21.2) -- (39.9,-36.6);
	\fill [black] (39.9,-36.6) -- (40.4,-35.8) -- (39.4,-35.8);
	\draw [black] (22.24,-20.17) -- (13.56,-27.73);
	\fill [black] (13.56,-27.73) -- (14.49,-27.58) -- (13.84,-26.83);
	\draw (16.89,-23.46) node [above] {$1$};
	\draw [black] (27.073,-19.74) arc (56.22581:15.25374:29.801);
	\fill [black] (27.07,-19.74) -- (27.46,-20.6) -- (28.02,-19.77);
	\draw (35.28,-25.73) node [right] {$1$};
	\draw [black] (37.273,-38.153) arc (-121.95795:-166.56249:27.601);
	\fill [black] (37.27,-38.15) -- (36.86,-37.31) -- (36.33,-38.15);
	\end{tikzpicture}
\end{center}

Максимальный поток $1+4=5$

(ii)

Исходя из последнего остаточного графа минимальный разрез это $\{s\}:\{-1,-4,-4,3,t\}$

(iii)
Задача поиска подмножества проектов с максимальной суммарной прибылью эквивалентна задаче поиска максимального потока в сети, в модификации где у вершин тоже есть веса





























\end{document} % конец документа