\documentclass[a4paper,12pt]{article} % добавить leqno в [] для нумерации слева

%%% Работа с русским языком
\usepackage{cmap}					% поиск в PDF
\usepackage{mathtext} 				% русские буквы в формулах
\usepackage[T2A]{fontenc}			% кодировка
\usepackage[utf8]{inputenc}			% кодировка исходного текста
\usepackage[english,russian]{babel}	% локализация и переносы

%%% Дополнительная работа с математикой
\usepackage{amsmath,amsfonts,amssymb,amsthm,mathtools} % AMS
\usepackage{icomma} % "Умная" запятая: $0,2$ --- число, $0, 2$ --- перечисление

%% Номера формул
\mathtoolsset{showonlyrefs=true} % Показывать номера только у тех формул, на которые есть \eqref{} в тексте.

%% Раисунки (автоматы)
\usepackage{tikz}

%% Шрифты
\usepackage{euscript}	 % Шрифт Евклид
\usepackage{mathrsfs} % Красивый матшрифт

%% Свои команды
\DeclareMathOperator{\sgn}{\mathop{sgn}}
\usepackage{ dsfont }

%% Перенос знаков в формулах (по Львовскому)
\newcommand*{\hm}[1]{#1\nobreak\discretionary{}
	{\hbox{$\mathsurround=0pt #1$}}{}}

%%% Заголовок
\author{771 группа, Христолюбов Максим}
\title{Домашнее задание по АМВ}
\date{\today}

\begin{document} % конец преамбулы, начало документа
	
	\maketitle

\section{Задача 1}
\subsection{(iii)}
\hspace{5mm}
$A_0=A_1=A_2=A_3=A_4=A_6=0$, $A_5=1$ из решений соответствующих уравнений.

Покажем, что $A_n=A_{n-6}+1$.

Как пример рассмотрим число $19$. При разложении $19=2\cdot 2 + 2\cdot 6 + 1\cdot 3=(2\cdot 2 + 2\cdot 6) + 1\cdot 3=8\cdot 2 + 1\cdot 3$ кол-во двоек максимально. Уменьшим кол-во двоек в разложении, но чтобы их было как можно больше, тогда $19=2\cdot 2 + 2\cdot 6 + 1\cdot 3=(2\cdot 2+ 1\cdot 6) + (1\cdot 6 + 1\cdot 3)=5\cdot 2+3\cdot 3$. Следующее разложение можно получить отправив всю часть, кратную $6$ к тройкам: $19=2\cdot 2 + 2\cdot 6 + 1\cdot 3=2\cdot 2+(2\cdot 6 + 1\cdot 3)=2\cdot 2 + 5\cdot 3$. Больше решений нет, так как нарушалось бы делимость $19$. То же самое можно проделать с любым $n$~---~фактически, здесь $n$ делилось на $6$ с остатком и раскладывалось в сумму двоек и троек, но так чтобы в разложении всегда присутствовали и $2$, и $3$, а всевозможные решения получались в зависимости от того сколько шестерок сгруппировать с $2$, а сколько с $3$. 

Пусть $n=a\cdot 2 + k\cdot 6+ b\cdot 3$, причем, $a$ и $b$ - минимальные натуральные из всех возможных. Тогда $n+6=a\cdot 2 + (k+1)\cdot 6+ b\cdot 3$. В случае с $n$ существовало $k$ вариантов как распределить $6k$ по двойкам и тройкам, в случае же $n+6$~---~$k+1$ вариант, значит $A_{n+6}=A_n+1$.

Из начальных условий $A_{6n+1}=A_{6n+2}=A_{6n+3}=A_{6n+4}=A_{6n+5}-1=A_{6n+6}=n$.

\subsection{(ii)}
\hspace{5mm}
$A_n=A_{n-6}+1=A_{n-12}+2=\ldots=A_{n-6k}+k=\Theta(k)$, где $k=[\frac{n}{6}]$

$A_n=\Theta(n)$

\subsection{(i)}
\hspace{5mm}

$F(x)=0+0x+0x^2+0x^3+0x^4+x^5+0x^6+x^7+x^8+\ldots+A_kx^k+\ldots
=0+\sum\limits_{k=0}^{\infty}A_{6k+1}x^{6k+1}+\sum\limits_{k=0}^{\infty}A_{6k+2}x^{6k+2}+\sum\limits_{k=0}^{\infty}A_{6k+3}x^{6k+3}+\sum\limits_{k=0}^{\infty}A_{6k+4}x^{6k+4}+\sum\limits_{k=0}^{\infty}A_{6k+5}x^{6k+5}+\sum\limits_{k=0}^{\infty}A_{6k+6}x^{6k+6}
=\sum\limits_{k=0}^{\infty}A_{6k+1}x^{6k+1}+\sum\limits_{k=0}^{\infty}A_{6k+1}x^{6k+2}+\sum\limits_{k=0}^{\infty}A_{6k+1}x^{6k+3}+\sum\limits_{k=0}^{\infty}A_{6k+1}x^{6k+4}+\sum\limits_{k=0}^{\infty}(A_{6k+1}+1)x^{6k+5}+\sum\limits_{k=0}^{\infty}A_{6k+1}x^{6k+6}
=(x+x^2\ldots+x^6)\sum\limits_{k=0}^{\infty}kx^{6k}+\sum\limits_{k=0}^{\infty}x^{6k+5}$

$\sum\limits_{k=0}^{\infty}x^{6k}=\frac{1}{1-x^6}$

$\sum\limits_{k=1}^{\infty}6kx^{6k-1}=\frac{6x^5}{(1-x^6)^2}$

$\sum\limits_{k=0}^{\infty}kx^{6k}=\sum\limits_{k=1}^{\infty}kx^{6k}=\frac{x^6}{(1-x^6)^2}$


$F(x)=\frac{x(1-x^6)}{1-x}\cdot\frac{x^6}{(1-x^6)^2}+\frac{x^5}{1-x^6}=\frac{x^7}{(1-x)(1-x^6)}+\frac{x^5}{1-x^6}=\frac{x^7-x^6+x^5}{(1-x)(1-x^6)}$

\section{Задача 2}

\subsection{(i)}
\hspace{5mm}
Пусть для определенности $y_{i-1}\geq x_{i-1}$, тогда $y_{i}=y_{i-1}\mod x_{i-1}<x_{i-1}=x_i$.

Рассмотрим случай, когда $x_{i-1}\leq \frac{y_{i-1}}{2}$. Значит,

$s_{i-1}=x_{i-1}+y_{i-1}\geq x_{i-1}+2x_{i-1}= 3x_{i-1}$

$s_{i}=x_{i}+y_{i}\leq x_{i}+x_{i}=2x_{i-1}\leq \frac{2}{3}s_{i-1}$

Если же $\frac{y_{i-1}}{2}\leq x_{i-1}$, то $y_i=y_{i-1}-x_{i-1}$

$s_{i-1}=x_{i-1}+y_{i-1}\geq \frac{y_{i-1}}{2}+y_{i-1}=\frac{3}{2}y_{i-1}$

$s_i=x_{i}+y_{i}=x_{i-1}+(y_{i-1}-x_{i-1})=y_{i-1}\leq \frac{2}{3}s_{i-1}$

В любом случае неравенство выполняется

\subsection{(ii)}
\hspace{5mm}
$\gcd(F_{m+2},F_{m+1})=\gcd(F_{m+1}+F_{m},F_{m+1})=\gcd(F_{m},F_{m+1})=\ldots =\gcd(F_2,F_1)=\gcd(1,1)=1$

\section{Задача 3}
\hspace{5mm}
$G_3(k)\leq 4G_3(\lceil{\frac{k}{4}}\rceil)+4\lceil{\frac{k}{4}}\rceil^3 $ $(u_i=\lceil{\frac{k}{4}}\rceil)$

$G_3(k)\geq 4G_3(\lfloor{\frac{k}{4}}\rfloor)+4\lfloor{\frac{k}{4}}\rfloor^3 $ $(u_i=\lfloor{\frac{k}{4}}\rfloor)$

$G_3(k)=\Theta(F)$, $F(k)=4F(\frac{k}{4})+4(\frac{k}{4})^3$

По основное теореме $F(k)=\Theta(k^3)$. Значит $G_3(k)=\Theta(k^3)$.

\section{Задача 4}
\subsection{(i)}
\hspace{5mm}
Условие задачи~---~на каждом $k$-ом шагу записи правильной скобочной последовательности $L(k)>R(k)$. Это значит, что первая скобка~---~обязательно открывающаяся, а последняя зарывающаяся, а все то что между ними представляет из себя произвольную правильную скобочную последовательность. Значит нужно посчитать кол-во правильных скобочных последовательностей из $n-2$ скобок и является частным случаем задачи о путях Дика (из курса АЛКТГ). Всего их $C_{n-2}^{\frac{n-2}{2}}-C_{n-2}^{\frac{n-2}{2}+1}=C_{n-2}^{\frac{n-2}{2}}-C_{n-2}^{\frac{n}{2}}$

\subsection{(ii)}
\hspace{5mm}
Кол-во скобочных последовательностей длины $4n+2$ - это $2n+1$-ое число Каталана. 

$Cat(x)=\frac{1-\sqrt{1-4x}}{2x}=c_0+c_1x+c_2x^2+\ldots$

Нужно найти сумму ряда

$S=BR_{2}+BR_6x+\ldots+BR_{4k+2}x^k+\ldots=c_{1}+c_3x+\ldots+c_{2k+1}x^k+\ldots$

$Cat(x)-Cat(-x)=c_0+c_1x+c_2x^2+c_3x^3+\ldots-c_0+c_1x-c_2x^2+c_3x^3+\ldots=2c_1x+2c_3x^3+\ldots=\frac{1-\sqrt{1-4x}}{2x}-\frac{1-\sqrt{1+4x}}{-2x}=\frac{2-\sqrt{1-4x}-\sqrt{1+4x}}{2x}$

$F(x)=\frac{Cat(x)-Cat(-x)}{2x}=c_1+c_3x^2+\ldots=\frac{2-\sqrt{1-4x}-\sqrt{1+4x}}{4x^2}$

$S=c_1+c_3x+\ldots=F(\sqrt{x})=\frac{2-\sqrt{1-4\sqrt{x}}-\sqrt{1+4\sqrt{x}}}{4x}$


\section{Задача 5}
\hspace{5mm}
$\sim$~---~ эквивалентность при $n\rightarrow\infty$, то есть предел отношения функций слева и справа стремится к константе. Округление, $-5$ и $-\log{\frac{3}{2}}$ из знаменателя можно выбросить, так как они не влияют на асимптотику при $n\rightarrow\infty$. Здесь имеет смысл рассматривать данные соотношения при $n\rightarrow\infty$, так как из определения предела следует, выполнение соотношений, начиная с некоторого $N$, а остальных $n$ конечное число, а значит взяв достаточную константу $C$, можно добится выполнения определения $\Theta$ ($cn\leq T(n)\leq Cn$).

$T(n)=3T(\lceil\frac{n}{\sqrt{3}}\rceil-5)+\frac{n^3}{\log{n}}
\sim3T(\frac{n}{\sqrt{3}})+\frac{n^3}{\log{n}}\sim
3(3T(\frac{n}{3})+\frac{n^3}{3^{\frac{3}{2}}(\log{n}-\log{3/2})})+\frac{n^3}{\log{n}}\sim
9T(\frac{n}{3})+({\frac{1}{3}})^{\frac{1}{2}}\frac{n^3}{\log{n}}+\frac{n^3}{\log{n}})\sim\ldots\sim
(3^{k}T(\frac{n}{3^k})+\frac{n^3}{\log{n}}\cdot (1+({\frac{1}{3}})^{\frac{1}{2}}+\ldots+({\frac{1}{3}})^{\frac{1}{2}k})\sim
(3^k T(1)+\frac{n^3}{\log{n}}\cdot \Theta(({\frac{1}{\sqrt{3}}})^{k})$, где $k=log_{\sqrt{3}}{n}$.

$T(n)=\Theta(3^{log_{\sqrt{3}}{n}}+\frac{n^3}{\log{n}}\cdot ({\frac{1}{\sqrt{3}}})^{log_{\sqrt{3}}{n}})=\Theta(n^2+\frac{n^3}{n\log{n}})=\Theta(\frac{n^2}{\log{n}})$

\section{Задача 6}
\hspace{5mm}
Запишем рекурентную формулу для сложности алгоритма. Поскольку на каждом шаге нужно проходится по всему массиву за $\Theta(n)$, искать медиану медиан этим же алгоритмом за $T(\frac{n}{4})$ и рекурсивно рассматривать подмассив, в котором находится медиана, размером $n-\frac{3}{8}n=\frac{5}{8}n$ в лучшем и размером $n-\frac{2}{8}n=\frac{3}{4}n$ в худшем случае, то в худшем случае 

$T(n)=T(\frac{n}{4})+T(\frac{3n}{4})+\Theta(n)=\Theta(n\log{n})$, т.~к. $\frac{n}{4}+\frac{3n}{4}=n$ по теореме из курса Алгоритмов, которая гласит, что если $T(n)=\sum\limits_{i=1}^{r}T(\frac{n}{b_i})+\Theta(n)$, где $\sum\limits_{i=1}^{r}\frac{n}{b_i}=n$, то $T(n)=\Theta(n\log n)$, а если $\sum\limits_{i=1}^{r}\frac{n}{b_i}<n$, то $T(n)=\Theta(n)$.

\section{Задача 7}
\hspace{5mm}
Если вместо каждого вызова функции расписывать соответствующий $S$ в формуле и расписывать до $n=100$, то в итоге получится $S(n)=X(n)\cdot S(100)$, где $X(n)$~---~кол-во вызовов функции. По скорости возрастания $X(n)$ и $S(n)$ эквивалентны. Для того чтобы оценить $X(n)$ можно решить характеристическое уравнение и общую формулу для $S(n)$ и через нее оценить $X(n)$.

$\lambda^3-\lambda^2-1=0$

$\lambda=1,47$ приблизительно.

Так как остальные корни комплексные и проявляются в формуле общего члена как синус и косинус, умноженные на константы, то при больших $n$ не вносят вклад в значение и  $S(n)~C\cdot 1,47^n$. 

Можно оценить и подругому: у функции $F(n)=F(n-1)+F(n-2)$ будет больше вызовов, а это рекурента чисел Фибоначчи,причем $F(100)>100$. Значит, $X(10^{12})<F(10^{12})<1,62^{10^{12}}(\phi=1,62)$. Если же рассмотреть $B(n)=B(n-2)+B(n-3)$, то расписываться она будет так же как и числа Фибоначчи:

$F(n)=2F(n-2)+F(n-3)$, $B(n)=2B(n-3)+B(n-4)$ и т.~п. 

Отличие только в том, что формула чисел Фибоначчи Находится в таком виде при расписывании до $n\rightarrow n-2$, а $B$ - до $n \rightarrow n-3$. Значит, итераций во втором случае будет в $1,5$ раза меньше, значит $X(10^{12})>1,62^{\frac{2}{3}10^{12}}=1,38^{10^{12}}$

Итак, $1,38^{10^{12}}<X(10^{12})<1,62^{10^{12}}$.

\section{Задача 8}
\hspace{5mm}


$n-\sqrt{n}>\sqrt{n}$ для всех больших $n$, значит самая длинная ветвь дерева рекурсии будет та в которой на каждом шаге будет расписываться $T(n-\sqrt{n})$. По асимптотике дерева рекурсии $T(n)\sim S(n), S(n)=S(n-\sqrt{n})$, т.~к. с каждой итерацией из некого текущего $n$ будет вычитаться $\sqrt{n}<\sqrt{n_0}$, где $n_0$~---~начальное $n$, по которому оценивается асимптотика. Если бы на каждой итерации вычиталось $\sqrt{n_0}$, то дерево было высотой $\frac{n_0}{\sqrt{n_0}}=\sqrt{n_0}$, высота дерева рекурсии $O(\sqrt{n})$

$S(2^k)=S(2^k-2^{\frac{k}{2}})=S(2^k-2^{\frac{k}{2}}-\sqrt{2^k-2^{\frac{k}{2}}})=S(2^k-2^{\frac{k}{2}}-2^{\frac{k}{2}}\sqrt{1-2^{\frac{-k}{2}}})=S(2^k-2\cdot 2^{\frac{k}{2}}+\frac{1}{2}(\frac{1}{2})^{\frac{k}{2}}-\ldots)$

Значит, высота дерева будет больше, чем, если бы $S(2^k)=S(2^k-2\cdot 2^{\frac{k}{2}}=\ldots=S(2^k-p\cdot 2^{\frac{k}{2}})$, чья высота 

$2^k-p\cdot 2^{\frac{k}{2}}=0$, $p=\frac{2^k}{2^\frac{k}{2}}=2^\frac{k}{2}$, $\Theta(2^\frac{k}{2})$

То есть $S(2^k)=\Omega(2^\frac{k}{2})$, $S(n)=\Omega(\sqrt n)=O(\sqrt n)=\Theta(\sqrt n)$ и $T(n)=\Theta(\sqrt{n})$
	

\section{Задача 9}
\hspace{5mm}
$T(n)=nT(\frac{n}{2})+O(n)=n(\frac{n}{2}T(\frac{n}{4})+O(\frac{n}{2}))+O(n)=\ldots
=n\cdot\frac{n}{2}\cdot\frac{n}{4}\cdots\frac{n}{2^{k-1}}T(1)+O(n)+\ldots+O(n^k)=
\frac{n^{\log_2{n}}}{2^{1+\ldots+\log_2{n}-1}}T(1)+O(n^{\log_2{n}})=
\frac{n^{\log_2{n}}}{2^{\frac{1}{2}\log_2{n}(\log_2{n}-1)}}T(1)+O(n^{\log_2{n}})=
\frac{n^{\log_2{n}}}{2^{\frac{1}{2}\log_2^2{n}}}\cdot 2^{\frac{\log_2{n}}{2}}+O(n^{\log_2{n}})=
n^{\log_2{n}}\cdot 2^{-\log_2^2{n^{\frac{1}{\sqrt{2}}}}}\cdot\sqrt{n}+O(n^{\log_2{n}})=
n^{\log_2{n}}\cdot (n^{\frac{1}{\sqrt{2}}})^{-\log_2{n^{\frac{1}{\sqrt{2}}}}}\cdot\sqrt{n}+O(n^{\log_2{n}})=
n^{\log_2{n}}\cdot n^{-\frac{1}{2}\log_2{n}}\cdot\sqrt{n}+O(n^{\log_2{n}})=
\Theta({\sqrt{n}^{\log_2{n}}}\sqrt{n})=\Theta(n^{\frac{1}{2}\log_2{2n}})$












\end{document} % конец документа