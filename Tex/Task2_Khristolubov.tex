\documentclass[a4paper,12pt]{article} % добавить leqno в [] для нумерации слева

%%% Работа с русским языком
\usepackage{cmap}					% поиск в PDF
\usepackage{mathtext} 				% русские буквы в формулах
\usepackage[T2A]{fontenc}			% кодировка
\usepackage[utf8]{inputenc}			% кодировка исходного текста
\usepackage[english,russian]{babel}	% локализация и переносы

%%% Дополнительная работа с математикой
\usepackage{amsmath,amsfonts,amssymb,amsthm,mathtools} % AMS
\usepackage{icomma} % "Умная" запятая: $0,2$ --- число, $0, 2$ --- перечисление

%% Номера формул
\mathtoolsset{showonlyrefs=true} % Показывать номера только у тех формул, на которые есть \eqref{} в тексте.

%% Раисунки (автоматы)
\usepackage{tikz}

%% Шрифты
\usepackage{euscript}	 % Шрифт Евклид
\usepackage{mathrsfs} % Красивый матшрифт

%% Свои команды
\DeclareMathOperator{\sgn}{\mathop{sgn}}
\usepackage{ dsfont }

%% Перенос знаков в формулах (по Львовскому)
\newcommand*{\hm}[1]{#1\nobreak\discretionary{}
	{\hbox{$\mathsurround=0pt #1$}}{}}

%%% Заголовок
\author{771 группа, Христолюбов Максим}
\title{Домашнее задание по АМВ}
\date{\today}

\begin{document} % конец преамбулы, начало документа
	
	\maketitle

\section{Задача 1}
\subsection{(i)}
\hspace{5mm}
Если использовать функцию $f(x)=x$, $x$~---~слово языка, то $\forall x\in A \rightarrow f(x)\in A$, значит по определению полиномиальной сходимости рефлексивность выполняется.

Если $A\leq_p B$ и $B\leq_p C$, тогда $\exists$ соответствующие $f(x)$ и $g(x)$, переводящие слово $x\in A$ в $f(x)\in B$, а потом в $g(f(x))\in C$ за полиномиальное время. Т.~к. $\phi(x)=g(f(x))$~---~полином, как композиция полиномов, то $A\leq_p C$.
 

\subsection{(ii)}
\hspace{5mm}
Если $B\in \mathcal{P}$ и $A\leq_p B$, тогда можно определить принадлежность $x$ языку $A$ так: вычислить за полиномиальное время $f(x)\in B$ и определить за полиномиальное время принадлежность к $B$ ($M(f(x))$ работает за полином), а т.~к. из определения сходимость $x\in A \Leftrightarrow f(x)\in B$, то и определить принадлежность $x$ языку $A$ за полиномиальное время. Значит, $A\in \mathcal{P}$

\subsection{(iii)}
\hspace{5mm}
Если $B\in \mathcal{NP}$ и $A\leq_p B$, тогда можно определить принадлежность $x$ языку $A$ так: вычислить за полиномиальное время $f(x)\in B$ и определить за полиномиальное время (на недерминированной машине Тьюринга) принадлежность к $B$ ($M(f(x))$ работает за полином на НМТ), а т.~к. из определения сходимость $x\in A \Leftrightarrow f(x)\in B$, то и определить принадлежность $x$ языку $A$ за полиномиальное на НМТ время. Значит, $A\in \mathcal{NP}$


\section{Задача 2}
\subsection{(i)}
\hspace{5mm}
Можно проверять всевозможные тройки из всех $n$ вершин графа, которых всего не более $n^3$, проверка происходит за полиномиальное время, так как можно быстро найти соответствующую клетку в матрице смежности. Проверку на двудольность можно проверить перебирая вершины и распределяя их в $2$ группы, и посмотреть получить ли их распределить в $2$ группы, в каждой из которых вершины не соединены. Значит, определить принадлежность графа языку за полиномиальное от кол-ва вершин время, а длина слова~---~размер матрицы смежности~---~полином от $n$. То есть язык лежит в $\mathcal{P}$.

\subsection{(ii)}
\hspace{5mm}
Несвязность и наличие циклов проверяется обходом в глубину, который работает полиномиально от кол-ва вершин, а значит полиномиально и от длины записи таблицы смежности, поэтому язык лежит в $\mathcal{P}$.

\subsection{(iii)}
\hspace{5mm}
Все такие подматрицы перебираются за полиномиальное от $n$ время и их проверка на выполнение условие тоже займет полином от $n$ времени, значит, принадлежность языку можно определить за полином от длины записи слова (матрицы), и язык лежит в $\mathcal{P}$.

\section{Задача 3}
\subsection{(i)}
\hspace{5mm}
При занулении первого столбца методом Гаусса, коэффициенты $a_{1i}^0$ в первой строчке умножаются на $a_{j1}^0$ и делятся $a_{11}^0$ получается $\frac{a_{j1}^0 a_{1i}^0}{a_{11}^0}$. После этого одна строка вычитается из другой, вычисляется $a_{ji}^1=a_{ji}^0-\frac{a_{j1}^0a_{1i}^0}{a_{11}^0}=\frac{a_{ji}^0a_{11}^0-a_{j1}^0a_{1i}^0}{a_{11}^0}$, в худшем случае числитель результата~---~порядка $2h^2$, знаменатель~---~$h$ у всех чисел в матрице, кроме первой строчки. 

На следующем шаге коэффициенты $a_{2i}$ в первой строчке умножаются на $a_{j2}$ и делятся $a_{22}$ получается $\frac{a_{j2}^1a_{2i}^1}{a_{22}^1}=\frac{(a_{j2}^0a_{11}^0-a_{j1}^0a_{12}^0)(a_{2i}^0a_{11}^0-a_{21}^0a_{1i}^0)a_{11}^0}{a_{11}^0a_{11}^0(a_{22}^0a_{11}^0-a_{21}^0a_{12}^0)}$. После этого одна строка вычитается из другой, вычисляется $a_{ji}^2=a_{ji}^1-\frac{a_{j1}^1a_{1i}^1}{a_{22}^1}=\frac{a_{ji}^1a_{22}^1-a_{j1}^1a_{1i}^1}{a_{22}^1}$, числитель~---~$8h^4\cdot h=8h^5$, знаменатель~---~$2h^2\cdot h^2=2h^4$ у всех чисел в матрице, кроме первой строчки. Вообще, если на предыдущем шаге у чисел в матрице числитель был пропорционален $bh^k$, а знаменатель $ch^p$, то на следующем шаге числитель~---~$2b^2h^{2k}\cdot ch^p$, а знаменатель $bh^k\cdot c^2h^{2p}$. Что означает, что на каждом шаге в худшем случае числитель и знаменатель увеличиваются как минимум в квадрат. После $\min(n,m)-1$ итераций, которые нужны для диагонализации матрицы размеры числителя и знаменателя будут не менее $h^{2^{(\min(m,n)-1)}}$ и $h^{2^{(\min(m,n)-1)-1}}$ соответственно, а длинны их записи $\log{h^{2^{(\min(m,n)-1)}}}$ и $\log{h^{2^{(\min(m,n)-1)-1}}}$, что $\Theta(2^{\min(m,n)})$ и не является полиномиальной оценкой.

\subsection{(ii)}
\hspace{5mm}
Так как при вычислении методом Гаусса $a_{ij}^{(k)}=\frac{det(D_{ij}^{(k)})}{det(D^{(k)})}$, то из формулы детерминанта коэффициенты матрицы при преобразовании методом Гаусса будут $O(h^{k}k)$, где $k=\min(m,n)$. Их умножение за $O(\log^2{h^k})=O(k^2\log^2{h})$, а кроме того их нужно сокращать алгоритмом Евклида за $\Theta(k\log h)$, то есть $O(k^3\log^3 h)$. На всех $k$ шагах диагонализация произойдут за $O(k^3\log^3 h \cdot n \cdot k)$ действий. Дальнейшее вычисление корней произойдет за меньшее кол-во умножений этих чисел Значит сложность $O(n(\min(m,n))^4\log^3 h)$.

\section{Задача 4}
\hspace{5mm}
Если $L\in \mathcal{P}$, то существует алгоритм $A(x)$ определяющий принадлежность языку за полиномиальное время $t(|x|)$. Построим алгоритм $A^*(x)$ для проверки принадлежности языку $L^*$. Заведем массив индексов концов слов из $L$, изначально $e=\{0\}$. Будем перебирать всевозможные подслова $x_1\ldots x_i$ и проверять алгоритмом $A$ их принадлежность $L$, а так же заносить их индексы в $e$. На следующей итерации переберем всевозможные слова с началом в $e_k+1$ и концом во всевозможных позициях $i$. Итерации будут продолжаться пока в $e$ не перестанут появляться новые позиции. Таким образом, в $e$ будут концы из всевозможных цепочек слов из $L$, конкатенация которых принадлежит префиксу $x$. Поэтому $x\in L^*$ тогда и только тогда, когда в $e$ будет $|x|$. Всего проверок на принадлежность $L$ будет не больше, чем подслов в $x$, не больше чем $|x|^2$, значит проверка займет не больше, чем $|x|^2t(|x|)$~---~полиномиальное время.

С замыканием $L^*,L\in \mathcal{NP}$ можно сделать то же самое. В качестве сертификата можно взять $s^*=\{s(x_i\ldots x_j)|x_i\ldots x_j~-~подслово\ x,\newline s~-~сертификат\ для\ алгоритма\ A\ проверки\ принадлежности\ к\ L\}$ и использовать их для определения принадлежности подслов языку $L$, поэтому с этим сертификатом алгоритм будет работать $|x|^2t(|x|)$.

\section{Задача 5}
\hspace{5mm}
Для проверки можно воспользоваться модифицированным методом Гаусса и диагонализировать расширенную матрицу системы. Если будет получена строчка, в которой все коэффициенты при $x_i=0$, а $b_j\neq 0$, тогда эта система несовместна. Как показано в пункте $(ii)$ $3$ номера размер дробей будет полиномиальным от размера системы, значит все коэффициенты, на которые умножаются строки матрицы, чья линейная комбинация в итоге обращаются в ноль, имеют размер полиномиальный от размеров матрицы. Значит, в качестве сертификата $y$ можно взять эти коэффициенты, с которыми нужно взять строки матрицы, чтобы получить нулевую строку, причем их длина $y$ будет полиномиальной от размера матрицы. Проверка на то что эта линейная комбинация действительно дает нулевую строку, а $b_j\neq 0$, произойдет за полиномиальное время, значит язык в классе $\mathcal{NP}$.


\section{Задача 6}
\hspace{5mm}
Если вместе с парой $(N,M)$ на вход машины Тьюринга предоставить сертификат $d$, которой является делителем $N$ и $1<d<M$, то МТ нужно будет только проверить, что $d$ удовлетворяет условиям, а так как алгоритм Евклида и сравнение работает за полиномиальное время, то проверка пройдет за полиномиальное время, значит $L_{factor}\in\mathcal{NP}$.

С другой стороны если в качестве сертификата предоставить все разложение $N$ на множители $p_1^{\alpha_1}\ldots p_k^{\alpha_k}$. (в таком случае длина сертификата будет полиномиальна от длины $N$, так как всего делителей у числа не более $N$, а длины чисел не превосходят $\log{N}$). Проверив делением, что они делители $N$, а так же их произведение дает $N$ (для того, чтобы убедиться, что больше делителей нет), а так же сравнив эти делители с $M$ можно будет проверить существует ли такой делитель $d$, удовлетворяющий условию, а значит, за полиномиальное время определить принадлежность дополнению $L_{factor}$, то есть $L_{factor}\in co-\mathcal{NP}$.

\section{Задача 7}
\hspace{5mm}
ГП можно полиномиально свести к ГЦ с помощью $f(x)=x$~---~чтобы проверить принадлежность $x$ к ГП можно проверив $x\in ГЦ$. Если это так,то $x\in ГП$. Действительно, если в графе есть гамильтонов цикл, то выкинув из гамильтонова цикла одно ребро можно получить гамильтонов путь, значит $ГП\subseteq ГЦ$.

Если есть МТ, распознающая ГП за полиномиальное время построим алгоритм, проверяющий принадлежность к ГЦ за полиномиальное время. Если добавить к графу $2$ ребра соединяющие вершины $i$ и новую вершину, а так же вершину $j$ и другую новую вершину, то МТ, распознающая ГП, даст положительный ответ тога и только тогда, когда $i$ и $j$~---~вершина, которые являются началом и концом для некоторого гамильтонова пути в изначальном графе. Перебрав все $i,j$, принадлежащие графу, так можно составить список всех пар вершин, которые являются началом и концом некоторых гамильтоновых путей. Если какая-то пара соединена ребром, то в изначальном графе есть гамильтонов цикл, совпадающий с соответствующим гамильтоновым путем плюс это ребро. Реализовав этот алгоритм на МТ получиться полиномиально работающий МТ, распознающую ГЦ, построенный на основе МТ, полиномиально распознающей ГП.


\section{Задача 8}
\hspace{5mm}
Пусть длина входа $|РВ|+|w|$.

Построим по РВ НКА, воспользовавшись стандартными реализациями $|,*$ и конкатенации~---~на месте конкатенации переход к следующему блоку по эпсилон переходу, на месте объединения эпсилон переходы к блокам, входящим в объединение, а на месте замыкании Клини эпсилон переход в начало, иначе говоря алгоритм построения НКА по РВ из курса ТРЯП. Время преобразования, как и кол-во вершин в НКА будет полиномиально зависеть от длины РВ. Для проверки $w\notin L$ подадим на вход НКА $w$. Из-за наличия эпсилон переходов будет образовываться дерево возможных путей. Будем обходить это дерево в ширину, для этого придется хранить массив вершин-состояний НКА, в которых может находится НКА на данный момент. Размер этого массива не превышает размера всего графа $|V|$, то есть полиномиален от длины РВ. Перебирать этот массив нужно будет не более $|w|$, значит всего не более $|V||w|$ переходов, что не более $(|V|+|w|)^2$. Значит, алгоритм полиномиален.

*При решении советовался с Александром Жоговым.




\end{document} % конец документа