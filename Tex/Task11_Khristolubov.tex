

\documentclass[a4paper,12pt]{article} % добавить leqno в [] для нумерации слева

%%% Работа с русским языком
\usepackage{cmap}					% поиск в PDF
\usepackage{mathtext} 				% русские буквы в формулах
\usepackage[T2A]{fontenc}			% кодировка
\usepackage[utf8]{inputenc}			% кодировка исходного текста
\usepackage[english,russian]{babel}	% локализация и переносы

%%% Дополнительная работа с математикой
\usepackage{amsmath,amsfonts,amssymb,amsthm,mathtools} % AMS
\usepackage{icomma} % "Умная" запятая: $0,2$ --- число, $0, 2$ --- перечисление

%% Номера формул
\mathtoolsset{showonlyrefs=true} % Показывать номера только у тех формул, на которые есть \eqref{} в тексте.

%% Раисунки (автоматы)
\usepackage{tikz}

%% Шрифты
\usepackage{euscript}	 % Шрифт Евклид
\usepackage{mathrsfs} % Красивый матшрифт
\usepackage{ upgreek }

%% Свои команды
\DeclareMathOperator{\sgn}{\mathop{sgn}}
\usepackage{ dsfont }

%% Перенос знаков в формулах (по Львовскому)
\newcommand*{\hm}[1]{#1\nobreak\discretionary{}
	{\hbox{$\mathsurround=0pt #1$}}{}}

\usepackage[left=2cm,right=2cm,
top=2cm,bottom=2cm,bindingoffset=0cm]{geometry}

%%% Заголовок
\author{771 группа, Христолюбов Максим}
\title{Домашнее задание по АМВ}
\date{\today}

\begin{document} % конец преамбулы, начало документа
	
	\maketitle
	
\section{Задача 1}
\hspace{5mm}
Двойственной задачей будет минимизация $3a+5b$, с ограничениями $2a+b-c=1$, $a+3b-d=1$, $a,b,c,d\geq 0$. Решением двойственной задачи является $a=\frac{2}{5}$, $b=\frac{1}{5}$, при котором $3a+5b=\frac{11}{5}$. При решении прямой задачи $x=\frac{4}{5}$, $y=\frac{7}{5}$, $x+y=\frac{11}{5}$. Так как значения функций решений прямой и обратной задачи совпадают, то эти решения оптимальны.

\section{Задача 2}
\hspace{5mm}
Проблема работы стандартного алгоритма поиска максимального потока в такой сети с потерями в том, что ребра и значения мощности потока на них выбираются случайным образом (с помощью поиска в глубину). Поэтому на каждом из шаге алгоритм находит не оптимальный, с точки зрения потока с потерями, путь, и обращает ребра, не входящие в максимальный поток. \textbf{Если на каждом шаге выбора пути, по которому потечет поток, выбирать путь, который сохраняет максимальную часть потока, идущего по нему, из всех путей из $s$ до $t$}, тогда алгоритм будет работать на сети с потерями. Действительно, выбор пути в изначальной задаче ни на что не влияет, поэтому выбор пути с максимальным коэффициентом сохранности не нарушит корректности и мощность будет максимальна. Однако теперь получившийся поток будет "наилучшего" качества в том смысле, что вершины $t$ достигнет максимальная его часть. 

Часть потока, которая сохранится, определяется произведением $\epsilon_v$ вершин, которые входят в путь. Для того, чтобы найти путь с минимальными потерями можно использовать модификацию алгоритма Дейкстры поиска кратчайшего пути для графа с весами на вершинах, а не на ребрах. Тогда в вершинах нужно установить весами коэффициенты потерь. От того что тут произведение, а не сумма весов, ничего не измениться (можно взять логарифмы). Алгоритм Дейкстры для вершин с весами можно получить присвоив всем ребрам нулевые веса, а каждую вершину заменить на ребро с соответствующим весом, в которое с одной стороны входят все ребра, входящую в изначальную вершину, а с другой стороны выходят все ребра, исходящие из исходной вершины. 





\section{Задача 3}
\hspace{5mm}
При оптимальном решении для одной из этих точек (для $k$-ой) выполняется неравенство $|ax_k+by_k+c|\geq|ax_i+by_i+c|$, то есть от нее отклонений максимально. Если знать, что это за точка, то можно решить задачу линейного программирования по минимизации $|x_ka+y_kb+c|$ (что эквивалентно последовательному решению двух задач линейного программирования: по минимизации выражения $x_ka+y_kb+c$ с условием, что оно не меньше $0$, и по максимизации $x_ka+y_kb+c$ с условием, что оно не больше $0$), с условием, что $|x_ka+y_kb+c|\geq|x_ia+y_ib+c|$. Ее ответ как раз будет ответом на задачу задания. 

Если для какой из $7$ точек отклонение от оптимального решения максимально неизвестно, то можно просто перебрать $k=1..7$, а потом выбрать минимальное значение $|x_ka+y_kb+c|$. Таким образом, решение сводится к решению $7\time 2\time 2 = 28$ задач линейного программирования ($7$ точек, $2$ на раскрытие одного модуля, $2$ на другого).

\section{Задача 4}
\hspace{5mm}
Многогранник органичен $6$ плоскостями, которые задаются уравнениями $x_1=0, x_1=1, x_2=\epsilon x_1, x_2 = 1-\epsilon x_1, x_3=\epsilon x_2, x_3 = 1-\epsilon x_2$. Точки пересечения этих плоскостей можно выстроить в требуемом порядке: $(0,0,0)\rightarrow(1,\epsilon,\epsilon^2)\rightarrow(1,1-\epsilon,\epsilon-\epsilon*2)\rightarrow(0,1,\epsilon)\rightarrow(0,1,1-\epsilon)\rightarrow(1,1-\epsilon,1-\epsilon+\epsilon^2)\rightarrow(1,\epsilon,1-\epsilon^2)\rightarrow(0,0,1)$.

\section{Задача 5}
\hspace{5mm}
Покажем, что невозможно ситуация, когда обе системы либо не совместны одновременно, либо обе совместны, а значит, одна совместна тогда и только тогда, когда не совместна другая.

Если они обе не совместны, то можно взять $x=0$, $y>0$, удовлетворяющий $A^Ty\geq0$, и для них будет выполняться $Ax>b$, а значит, $(Ax)^Ty>b^Ty$. Тогда $0=x^TA^Ty>b^Ty\geq0$~---~противоречие.

Если они обе совместны, тогда, так как $y\geq0,b\geq Ax$, $b^Ty\geq (xA)^Ty$. А из $x>0,A^Ty\geq 0$ следует $x^TA^Ty\geq0$. Тогда $0>b^Ty\geq x^TA^Ty\geq 0$~---~противоречие.

\section{Задача 6}
\hspace{5mm}
Аналогично, покажем, что системы не могут быть одновременно совместны и не совместны.

Если обе не совместны, то возьмем $x\geq 0,y>0$, для них $Ax>0$, значит, $y^TAx\geq0$, так же $A^Ty\leq0$, значит, $x^TA^Ty\leq0$. Тогда $0<y^TAx=x^TAy\leq0$~---~противоречие.

Если обе системы совместны, то $y^TAx\leq0$, так же $x^TA^Ty>0$. Тогда $0\geq y^TAx=x^TA^Ty>0$~---~противоречие.	










\section{Задача 7}
\hspace{5mm}
К задаче $\left\{
\begin{array}{ccc}
c^Tx & \rightarrow & max \\
Ax & \leq & b \\
\end{array}\right.$ двойственной является задача $\left\{
\begin{array}{ccc}
b^Ty & \rightarrow & min \\
yA & = & c \\
y & \geq & 0 \\
\end{array}\right.$, а двойственной к ней $\left\{
\begin{array}{ccc}
(c^*)^Tz & \rightarrow & max \\
(A\ E)z & = & b^* \\
z & \geq & 0 \\
\end{array}\right.$, где $c^*$~---~вектор $c$, дополненный $0$, в тех компонентах, которые умножаются на неравенства из $y\geq 0$. Размерность $z$ не совпадает с размерностью $x$, так как условий в двойственной задаче (а их количество и равно количеству компонент $z$) к изначальной задаче больше, чем переменных в прямой задаче из-за $y\geq 0$.

Но если отдельно работать с неравенствами при нахождении двойственной к двойственной задаче, то получиться$\left\{
\begin{array}{ccc}
c^Tz & \rightarrow & max \\
Ax & \leq	 & b \\
\end{array}\right.$, что совпадает с исходной задачей.






\end{document} % конец документа
