\documentclass[a4paper,12pt]{article} % добавить leqno в [] для нумерации слева

%%% Работа с русским языком
\usepackage{cmap}					% поиск в PDF
\usepackage{mathtext} 				% русские буквы в формулах
\usepackage[T2A]{fontenc}			% кодировка
\usepackage[utf8]{inputenc}			% кодировка исходного текста
\usepackage[english,russian]{babel}	% локализация и переносы

%%% Дополнительная работа с математикой
\usepackage{amsmath,amsfonts,amssymb,amsthm,mathtools} % AMS
\usepackage{icomma} % "Умная" запятая: $0,2$ --- число, $0, 2$ --- перечисление

%% Номера формул
\mathtoolsset{showonlyrefs=true} % Показывать номера только у тех формул, на которые есть \eqref{} в тексте.

%% Раисунки (автоматы)
\usepackage{tikz}

%% Шрифты
\usepackage{euscript}	 % Шрифт Евклид
\usepackage{mathrsfs} % Красивый матшрифт
\usepackage{ upgreek }

%% Свои команды
\DeclareMathOperator{\sgn}{\mathop{sgn}}
\usepackage{ dsfont }

%% Перенос знаков в формулах (по Львовскому)
\newcommand*{\hm}[1]{#1\nobreak\discretionary{}
	{\hbox{$\mathsurround=0pt #1$}}{}}

%%% Заголовок
\author{771 группа, Христолюбов Максим}
\title{Домашнее задание по АМВ}
\date{\today}

\begin{document} % конец преамбулы, начало документа
	
	\maketitle
	

\section{Задача 1}
\hspace{5mm}

$(M_n(\omega))^{-1}M_n(\omega)=E$

Убедимся, что элемент с индексами $(j,k)$ $(M_n(\omega))^{-1}$ есть $\frac{\omega_n^{-kj}}{n}$, и этим покажем, что $(M_n(\omega))^{-1} = \dfrac{1}{n}M_n(\omega^{-1})$. Элемент $(i,j)$ матрицы $(M_n(\omega))^{-1}M_n(\omega)$ равен

$\sum\limits_{k=o}^{n-1}(\frac{\omega_n^{-ik}}{n})(\omega_n^{kj})=\frac{1}{n}\sum\limits_{k=o}^{n-1}\omega_n^{k(j-i)}$.

Если $i=j$, то все слагаемые равны $1$ и значение выражения тоже $1$. Если $i\neq j$, то при сложении слагаемых, которые являются комплексными числами, по правилу треугольника, получается правильный многоугольник (точнее $|i-j|$ многоугольников), и сумма соответственно равна $0$. Значит, элементы $(M_n(\omega))^{-1}$ действительно имеют такой вид.

Найдем элемент $(i,j)$ матрицы $(M_n(\omega))^{2}$:

$\sum\limits_{k=0}^{n-1}\omega_n^{ik}\omega_n^{kj}=\sum\limits_{k=0}^{n-1}\omega_n^{k(i+j)}$

Если $i+j=n$ или $i=j=0$, то сумма равна $n$, так как все слагаемые $1$. В остальных случаях сумма обратится в $0$. Возведем эту матрицу еще раз в квадрат. Так как только произведение $i$-ого вектор-столбцы на $i$-ый вектор-строку будет не $0$, то все элементы будут равны $0$, кроме диагональных, которые будут равны $n^3$. Значит, $(M_n(\omega))^{4}=n^3E$


\section{Задача 2}
\hspace{5mm}
Сначала посчитаем матрицы второго порядка, на их основе четвертого, потом восьмого

$M
\begin{pmatrix}
	2\\ 
	0\\
\end{pmatrix}=\begin{pmatrix}
2\\ 
2\\
\end{pmatrix}$, $M
\begin{pmatrix}
0\\ 
0\\
\end{pmatrix}=\begin{pmatrix}
0\\ 
0\\
\end{pmatrix}$, $M
\begin{pmatrix}
3\\ 
0\\
\end{pmatrix}=\begin{pmatrix}
3\\ 
3\\
\end{pmatrix}$, $M
\begin{pmatrix}
1\\ 
0\\
\end{pmatrix}=\begin{pmatrix}
1\\ 
1\\
\end{pmatrix}$

$M
\begin{pmatrix}
2\\  0\\ 0\\ 0\\
\end{pmatrix}=\begin{pmatrix}
2\\  2\\ 2\\ 2\\
\end{pmatrix}$, $M
\begin{pmatrix}
3\\  1\\ 0\\ 0\\
\end{pmatrix}=\begin{pmatrix}
4\\  3+\omega_4\\ 2\\ 3-\omega_4
\end{pmatrix}$

$M
\begin{pmatrix}
2\\  0\\ 0\\ 0\\ 3\\  1\\ 0\\ 0\\
\end{pmatrix}=\begin{pmatrix}
6\\  2+3\omega_8+\omega_8^3\\ 2+2\omega_8^2\\ 2+\omega_8+3\omega_8^3\\ -2\\  2-3\omega_8-\omega_8^3\\ 2-2\omega_8^2\\ 2-\omega_8-\omega_8^3\\
\end{pmatrix}$, аналогично $M
\begin{pmatrix}
2\\  0\\ 0\\ 0\\ 3\\  1\\ 0\\ 0\\
\end{pmatrix}=\begin{pmatrix}
8\\  2+3\omega_8+\omega_8^3\\ -1-3\omega_8^2\\ 2+3\omega_8-3\omega_8^3\\ 2\\  2+3\omega_8-3\omega_8^3\\ -1+3\omega_8^2\\ 2-3\omega_8-3\omega_8^3\\
\end{pmatrix}$

$A\cdot B =\begin{pmatrix}
	48\\  (2+3\omega_8+\omega_8^3)(2+3\omega_8+\omega_8^3)\\ (2+2\omega_8^2)(-1-3\omega_8^2)\\ (2+\omega_8+3\omega_8^3)(2+3\omega_8-3\omega_8^3)\\ -4\\  (2-3\omega_8-\omega_8^3)(2+3\omega_8-3\omega_8^3)\\ (2-2\omega_8^2)(-1+3\omega_8^2)\\ (2-\omega_8-\omega_8^3)(2-3\omega_8-3\omega_8^3)\\
\end{pmatrix}$

Выполняя сначала такие же действия для подвекторов размера $2$ и $4$, вычислим  $\begin{pmatrix}
4\\  6\\ 6\\ 17\\ 9\\  3\\ 3\\ 0\\
\end{pmatrix}$



\section{Задача 3}
\hspace{5mm}
Для перемножения произведения этих $n$ многочленов разобъем его на 2 произведения по $\frac{n}{2}$ многочленов и найдем рекурсивно с помощью этого алгоритма их коэффициенты. После этого перемножим эти $2$ многочлена с помощью БПФ за $O(n\log_2 n)$. Реккурента для этого алгоритма $T(n)=2T(\frac{n}{2})+O(n\log_2 n)$, значит, $T(n)=O(n\log_2^2 n)$.

\section{Задача 4}
\hspace{5mm}
$\begin{pmatrix}
1 & 8 & 4 & 2 \\
2 & 1 & 8 & 4 \\
4 & 2 & 1 & 8 \\
8 & 4 & 2 & 1 \\
\end{pmatrix}x=
\begin{pmatrix}
16 \\
8 \\
4 \\
2 \\
\end{pmatrix}$

Разложим циклическую матрицу

$\begin{pmatrix}
1 & 8 & 4 & 2 \\
2 & 1 & 8 & 4 \\
4 & 2 & 1 & 8 \\
8 & 4 & 2 & 1 \\
\end{pmatrix}=C=\frac{1}{n}V^*\Lambda V$

$\Lambda=diag(1+2+4+8,1+2i-4-8i,1-2+4-8,1-2i-4+8i)=diag(15,-3-6i,-5,-3+6i)$

$\Lambda^{-1}=diag(\frac{1}{15}, -\frac{1}{3(1+2i)}, -\frac{1}{5}, -\frac{1}{3(1-2i)})$

$V=\begin{pmatrix}
1 & 1 & 1 & 1 \\
1 & i & -1 & -i \\
1 & -1 & 1 & -1 \\
1 & -i & -1 & i \\
\end{pmatrix}$, $V^*=\begin{pmatrix}
1 & 1 & 1 & 1 \\
1 & -i & -1 & i \\
1 & -1 & 1 & -1 \\
1 & i & -1 & -i \\
\end{pmatrix}$

$x=\frac{1}{4}V(\Lambda^{-1}(V^{*}b))$

$V^{*}b=\begin{pmatrix}
30  \\
12-6i  \\
10  \\
12+6i  \\
\end{pmatrix}$, 

$\Lambda^{-1}(V^{*}b)=\begin{pmatrix}
\frac{1}{15}30  \\
-\frac{1}{3(1+2i)}(12-6i)  \\
-\frac{1}{5}10  \\
-\frac{1}{3(1-2i)}(12+6i)  \\
\end{pmatrix}
=\begin{pmatrix}
2  \\
-1,6+1,2i  \\
-2  \\
-1,6-1,2i \\
\end{pmatrix}$

$x=\frac{1}{4}V\begin{pmatrix}
2  \\
-1,6+1,2i  \\
-2  \\
-1,6-1,2i \\
\end{pmatrix}=\begin{pmatrix}
-\frac{4}{5}  \\
\frac{8}{5}  \\
\frac{4}{5}  \\
\frac{2}{5} \\
\end{pmatrix}$

\section{Задача 5}
\hspace{5mm}
Циркулянтная матрица раскладывается в $circ(x)=M^{-1}diag(FFT(x))M$, поэтому $FFT(x*y)=FFT(circ(x)y)=M circ(x)y = M(M^{-1}diag(FFT(x))M)y=diag(FFT(x))My=diag(FFT(x))FFT(y)=FFT(x)\times FFT(y)$ по Адамару.

\section{Задача 6}
\hspace{5mm}
*Решение взято у Михаила Сысака.

Собственными векторами циркулянтной матрицы являются векторы $v_k=(1,\omega_{n+1}^{k}\ldots\omega_{n+1}^{kn})^{T}$ $\forall k=0..n$. При умножении собственного вектора на циркулянтную матрицу из всех строк получившегося вектора можно вынести число $c_0+c_n\omega_{n+1}^k+\ldots+c_1\omega_{n+1}^{kn}$ и получить в векторе $v_k$, 
значит, $\lambda_k=c_0+c_n\omega_{n+1}^k+\ldots+c_1\omega_{n+1}^{kn}$, в это ни что иное, как произведение $(c_0,\ldots,c_n)$ на $k$-ую строчку в матрице Фурье. 
Значит, все $\lambda_k$ можно найти как значения $f(x)=c_0+c_1x+\ldots+c_nx^n$ в 
точках корнях из $1$ степени $n+1$ (которая степень $2$, а если нет добавить фиктивных точек до степени $2$) умножением матрицы Фурье на этот вектор коэффициентов.

$\begin{pmatrix}
	1  & 1& 1& 1\\
	1  & i& -1& -i\\
	1  & -1& 1& -1\\
	1  & -i& -1& i\\
\end{pmatrix}\begin{pmatrix}
1  \\
6  \\
4  \\
2  \\
\end{pmatrix}=\begin{pmatrix}
13  \\
-3+4i  \\
-3  \\
-3-4i  \\
\end{pmatrix}$

\section{Задача 7}
\hspace{5mm}
Построим эту процедуру. Составим и перемножим с самим собой с помощью БПФ многочлен $\sum\limits_{k\in A}x^k$. Степени $x$, при которых коэффициенты получившегося многочлена будут не $0$, и только они будут лежать во множестве $A+A$. Действительно, если такая степень $n$ получилась, то существуют $n_1$ и $n_2$ из $A$, что $n_1+n_2=n$, и для всех $n_1$ и $n_2$ из $A$ существует $n$~---~степень $x$ в получившемся полиноме, что $n_1+n_2=n$.

\section{Задача 8}
\hspace{5mm}
*Решение взято у Михаила Сысака.
\section{(i)}
\hspace{5mm}
Если вычислить массив $B_i=\sum\limits_{j=0}^{m-1}(p_j-t_{i+j})^2=\sum\limits_{j=0}^{m-1}p_j-2\sum\limits_{j=0}^{m-1}p_jt_{i+j}+\sum\limits_{j=0}^{m-1}t_{i+j}^2$. Перемножив за $O(n\log{n})$ многочлены $1+\ldots+x^{m-1}$ с $p_{m-1}^2+\ldots+p_0^2x^{m-1}$, $1+\ldots+x^{n-1}$ с $t_{0}^2+\ldots+t_{n-1}^2x^{n-1}$ и $t_0+\ldots+t_{n-1}x^{n-1}$ с $p_{m-1}+\ldots+p_0x^{m-1}$ можно найти значения сумм~---~коэффициент при $x^{m-1+i}$ в последнем произведении будет равен искомой средней сумме, то есть найдеными оказались все элементы $B$.

\section{(ii)}
\hspace{5mm}
Можно заменить всех джокеров на $0$, и при вычислении суммы каждый элемент домножать произведение $2$ рассматриваемых символов. Тогда слагаемое будет равно $0$ либо когда символы равны, либо когда один из них джокер. Теперь нужно вычислить значения трех сумм, каждое слагаемое которых состоит из $4$ множителей. Это можно сделать найдя произведение $4$ правильно подобранных многочленов за $)(n\log n)$.

\section{(iii)}
\hspace{5mm}
Можно покрыть текст подстроками длины $2m$, чтобы соседние из них пересекались по $m$ символам. На каждой из подстрок запустим описанные алгоритмы за $O(m\log{m})$, всего $\frac{n}{m}$ запусков. Тогда асимптотика будет $\frac{n}{m}O(m\log{m}=O(n\log{m}))$.





\section{Задача 9}
\hspace{5mm}
Вычислить $\{y_k\}_{k=0}^{n-1}$ можно за $O(n\log{n})$, после этого найдем суммированием за $O(n)$ $\sum\limits_{i=0}^{n-1}y_i$ и после сложим мнимую и действительную часть результата. Алгоритм работает за $O(n\log{n})=o(n^2)$.












\end{document} % конец документа