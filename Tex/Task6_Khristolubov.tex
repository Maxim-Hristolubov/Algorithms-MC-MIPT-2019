\documentclass[a4paper,12pt]{article} % добавить leqno в [] для нумерации слева

%%% Работа с русским языком
\usepackage{cmap}					% поиск в PDF
\usepackage{mathtext} 				% русские буквы в формулах
\usepackage[T2A]{fontenc}			% кодировка
\usepackage[utf8]{inputenc}			% кодировка исходного текста
\usepackage[english,russian]{babel}	% локализация и переносы

%%% Дополнительная работа с математикой
\usepackage{amsmath,amsfonts,amssymb,amsthm,mathtools} % AMS
\usepackage{icomma} % "Умная" запятая: $0,2$ --- число, $0, 2$ --- перечисление

%% Номера формул
\mathtoolsset{showonlyrefs=true} % Показывать номера только у тех формул, на которые есть \eqref{} в тексте.

%% Раисунки (автоматы)
\usepackage{tikz}

%% Шрифты
\usepackage{euscript}	 % Шрифт Евклид
\usepackage{mathrsfs} % Красивый матшрифт
\usepackage{ upgreek }

%% Свои команды
\DeclareMathOperator{\sgn}{\mathop{sgn}}
\usepackage{ dsfont }

%% Перенос знаков в формулах (по Львовскому)
\newcommand*{\hm}[1]{#1\nobreak\discretionary{}
	{\hbox{$\mathsurround=0pt #1$}}{}}

%%% Заголовок
\author{771 группа, Христолюбов Максим}
\title{Домашнее задание по АМВ}
\date{\today}

\begin{document} % конец преамбулы, начало документа
	
	\maketitle
	
*При решение некоторых номеров обращался к Михаилу Сысаку.

\section{Задача 1}
\hspace{5mm}

Для каждого языка из $\mathcal{RP}$ существует вероятностная МТ, принимающая слова этого языка. Пусть ДМТ будет работать так же к и ВМТ, а вместо использования случайных битов будет брать биты из предоставленного ей сертификата. Так как вероятность принятия ВМТ слова из языка не нулевая, то существует такой сертификат, что ДМТ примет его, а для слова не из языка не существует такого сертификата, посколько вероятность его принятия равна $0$. Причем длина сертификата полиномиальна, так как он состоит из всех используемых ВМТ битов, а из полиномиальной работы ВМТ следует полиномиальность используемых битов. Значит, существует ДМТ, распознающая этот язык с сертификатом, и этот язык $\mathcal{NP}$.

\section{Задача 2}
\hspace{5mm}

$2^n\geq |x-y|\geq p_1p_2\ldots p_k\geq n^k$, значит, $k\leq \frac{n}{\ln{n}}\ln{2}$.

Так как $0,99\frac{n}{\ln{n}}\leq\pi(n)\geq 1,01\frac{n}{\ln{n}}\ln{n}$, то $\pi(n,2n)\geq 1,98\frac{n}{\ln{2n}}-1,01\frac{n}{\ln{n}}$

$P\{выбрать\ простое\ число-делитель\}=\frac{k}{\pi(n,2n)}\leq \frac{n\ln{2}}{1,98\frac{n\ln{n}}{ln{n}+\ln{2}}-1,01\frac{n\ln{n}}{\ln{n}}}=\frac{\ln{2}}{1,98\frac{\ln{n}}{\ln{n}+\ln{2}}-1,01}\leq\frac{3}{4}$

$n\geq e^{30}$, что около 1 теробайта.

\section{Задача 3}
\subsection{(i)}
\hspace{5mm}
Заменим вероятность ошибки $\frac{1}{3}$ на число $\frac{1-\alpha}{2}$. Тогда можно применить этот алгоритм $n$ нечетное кол-во раз и смотреть какой ответ будет дан чаще и его выдавать. Вероятность ошибки результата будет 

$P=\sum\limits_{i=0}^{\frac{n-1}{2}}C_n^i(\frac{1-\alpha}{2})^{n-i}(\frac{1+\alpha}{2})^i\leq \sum\limits_{i=0}^{\frac{n-1}{2}}C_n^i(\frac{1-\alpha}{2})^{\frac{n+1}{2}}(\frac{1+\alpha}{2})^{\frac{n-1}{2}}
\leq (\frac{1-\alpha}{2})^{\frac{n+1}{2}}(\frac{1+\alpha}{2})^{\frac{n-1}{2}}\sum\limits_{i=0}^{\frac{n-1}{2}}C_n^i \leq (1-\alpha^2)^{\frac{n-1}{2}}(1-\alpha)\frac{1}{2^n}2^n=(1-\alpha^2)^{\frac{n-1}{2}}(1-\alpha)\rightarrow 0$ при $n\rightarrow\infty$

Значит, существует такое $n$, что вероятность ошибки меньше $\frac{1}{3}$. И каждый язык, удовлетворяющий определению с $\frac{1-\alpha}{2}$ удовлетворяет определению с $\frac{1}{3}$.

\subsection{(ii)}
\hspace{5mm}
Изменим алгоритм~---~в случае превышения полиномиального времени работы будем случайно выдавать $0$ или $1$ с вероятностью $\frac{1}{2}$. $p$ - вероятность, что будет превышено полиномиальное время работы, $\epsilon\leq\frac{1}{2}$ - вероятность ошибки изначального алгоритма. Тогда ошибка измененного алгоритма

$\epsilon'=\frac{1}{2}p+\epsilon(1-p)\leq\frac{1}{2}p+\frac{1}{2}(1-p)=\frac{1}{2}$.

Значит измененный алгоритм будет удовлетворять определению, так как работает полиномиальное время.

\section{Задача 4}
\subsection{(i)}
\hspace{5mm}
Вероятность совпадения компоненты в равенстве $A(Bx)=Cx$ $\frac{1}{N}$, а вероятность совпадения векторов $\frac{1}{N^n}<p$, $N>\frac{1}{p^{\frac{1}{n}}}$.

\subsection{(iv)}
\hspace{5mm}
$(ABx)^Tx=(Cx)^Tx$ и $(ABx)^Ty=(Cx)^Ty$~---~равенства двух многочленов степени $2$. По лемме Шварца-Зиппеля

$P\{ошибка\}\leq\frac{2}{N}<p$, $N>\frac{2}{p}$

\section{Задача 5}
\subsection{(i)}
\hspace{5mm}
Пусть кол-во ребер минимального разреза $e$, тогда из каждой вершин выходит не более $e$ ребер, иначе существовал бы разрез минимальнее. 

$P\{выбрать\ ребро\ из\ разреза\}=\frac{e}{E}=\frac{e}{\frac{1}{2}eV}=\frac{2}{V}$

\subsection{(ii)}
\hspace{5mm}
Алгоритм выдаст верный ответ, если по ходу работы не будет стянуто ни одно из ребер, его пересекающих. На первом шаге вероятность выбрать ребро не из разреза $\frac{n-2}{n}$, на втором $\frac{n-3}{n-1}$ и так далее, вероятность выдать верный ответ $\frac{n-2}{n}\frac{n-1}{n-3}\ldots{frac{1}{3}=\frac{2}{(n-1)n}}$

\subsection{(ii)}
\hspace{5mm}
Если повторять алгоритм $n^2$ раз

$P\{ошибки\}\leq\left( 1-\frac{2}{n(n-1)}\right)^{n^2}\leq \left( 1-\frac{2}{n^2}\right)^{n^2}\rightarrow e^{-2} < 0,15$

Значит, начиная с достаточно большого $n_0$ вероятность правильного ответа будет превышать $0,85$.

\section{Задача 6}
\hspace{5mm}
Заменив каждую дезъюнкцию $(a\vee b)$ на $(\overline{a}\rightarrow b)\wedge(\overline{b}\rightarrow a)$. Теперь построим граф на всех литералах, такой что ребро $(u,v)$ принадлежит графу, если в конъюнкцию входит $(u\rightarrow b)$. Покажем, что формула выполнима тогда и только тогда, когда для любой переменной $x$ нельзя достичь $x$ из $\overline{x}$ и $\overline{x}$ из $x$.

Пусть формула выполнима. Предположим, для $x$ можно достичь его из отрицания и наоборот. Если $x=0$ в выполняющем наборе, тогда в одной из импликаций, которая ведет от отрицания к переменной импликация не выполнена, что противоречит выполнимости. Аналогично, если $x=1$.

Обратно, все $x$, из которого можно достичь $\overline{x}$ обозначим его $x=0$. Из единичной вершины не может быть достижима нулевая вершина, так как тогда бы формула была б невыполнимой. Всем вершинам, достижимых из единичных, присвоим $1$. Это присваивание непротиворечиво, так как если бы $x$ и $\overline{x}$ были бы достижимы из $y=1$ это значило бы что они достижимы друг из друга, что не возможно. Остальным вершинам значения можно присвоить произвольно и получить выполняющий набор, значит, формула выполнима.

Преобразование импликаций, построение графа, поиск компонент сильной связности (для проверки достижимости каждой переменной ее отрицания и наоборот) потребует полиномиального времени, значит, $2-SAT\in\mathcal{P}$.


\section{Задача 7}
\subsection{(i)}
\hspace{5mm}
Доказательство по индукции. На $1$ шаге колода равновероятно перемешана. Если на $k$ шаге все карты под $n-1$ были равномерно перемешаны, тогда на $(K+1)$ шаге мы засовываем верхнюю карту в случайное место. Если засунули карту выше $n-1$ карты, то под ней ничего не изменилось. Если ниже $n-1$ карты, под которой $p-1$ карта, то фактически карта была вставленно в случайное из $p$ мест. То есть новая перестановка была составлена сначала выбором одного из $p$ мест, а потом заполнением оставшихся мест одной из случайных перестановок, чья вероятность $\frac{1}{(p-1)!}$, тогда вероятность новой перестановки из $p$ карт $\frac{1}{(p-1)!}\frac{1}{p}=\frac{1}{p!}$, то есть они равновероятны.


\subsection{(ii)}
\hspace{5mm}
Вставка в равновероятно перемешанную колоду эквивалента вставке в одно из случайных мест между картами, что как было показано ранее порождает равновероятную перестановку.

\subsection{(iii)}
\hspace{5mm}
Матожидание времени работы алгоритма найдем как сумму матожиданий времени работы на $n-1$ шагах работы алгоритма, где матожидание $k$-ого шага~---~это сколько нужно раз в среднем попытаться засунуть верхнюю карту колоды, в которой под $n-1$ картой $k$ карт, чтобы она вставилась под $n-1$ карту, и под ней оказалось $k+1$ карт. Вероятность засунуть карту под $n-1$ на $k$-ом шаге $\frac{k+1}{n}$, значит, матожидание времени работы $k$-ого шага равно $\frac{n}{k+1}$. Матожидание алгоритма:

$E=\sum\limits_{k=1}^{n-1}\frac{n}{k+1}$

\section{Д-1 из файла}
\subsection{(i)}
\hspace{5mm}
$T(n)=4T(\frac{n}{2})+4F(n,\frac{n}{2})+\Theta(n^2)$, где $F(n,k)$~---~время стягивание графа на $n$ вершинах и порогом $k$.

\subsection{(ii)}
\hspace{5mm}
$F(n,k)=\Theta(n(n-k))=\Theta(n^2)$, так как нужно избавиться от  $n-k$ вершин, которые в худшем случае соеденены со всеми остальными вершинами.

$T(n)=4T(\frac{n}{2})+\Theta(n^2)$

По мастер теореме $T(n)=\Theta(n^2\log{n})$.

\subsection{(ii)}
\hspace{5mm}
$\frac{dp}{dk}=p_{k+1}-p_k=-\frac{3}{8}p_k^2$, $p_0=1$~---~задача Коши, $p_k=\frac{8}{3k+8}$.



\end{document} % конец документа