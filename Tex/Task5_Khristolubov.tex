\documentclass[a4paper,12pt]{article} % добавить leqno в [] для нумерации слева

%%% Работа с русским языком
\usepackage{cmap}					% поиск в PDF
\usepackage{mathtext} 				% русские буквы в формулах
\usepackage[T2A]{fontenc}			% кодировка
\usepackage[utf8]{inputenc}			% кодировка исходного текста
\usepackage[english,russian]{babel}	% локализация и переносы

%%% Дополнительная работа с математикой
\usepackage{amsmath,amsfonts,amssymb,amsthm,mathtools} % AMS
\usepackage{icomma} % "Умная" запятая: $0,2$ --- число, $0, 2$ --- перечисление

%% Номера формул
\mathtoolsset{showonlyrefs=true} % Показывать номера только у тех формул, на которые есть \eqref{} в тексте.

%% Раисунки (автоматы)
\usepackage{tikz}

%% Шрифты
\usepackage{euscript}	 % Шрифт Евклид
\usepackage{mathrsfs} % Красивый матшрифт
\usepackage{ upgreek }

%% Свои команды
\DeclareMathOperator{\sgn}{\mathop{sgn}}
\usepackage{ dsfont }

%% Перенос знаков в формулах (по Львовскому)
\newcommand*{\hm}[1]{#1\nobreak\discretionary{}
	{\hbox{$\mathsurround=0pt #1$}}{}}

%%% Заголовок
\author{771 группа, Христолюбов Максим}
\title{Домашнее задание по АМВ}
\date{\today}

\begin{document} % конец преамбулы, начало документа
	
	\maketitle

\section{Задача 1}
\hspace{5mm}
У полоски $2\times n$ $n$-ые 2 клетки могут покрываться либо черным квадратом, либо белым квадратом, либо серым прямоугольником, занимающим эти 2 $n$-ые клетки, либо двумя серыми прямоугольниками, каждый покрывающий одну из этих клеток. В $1$, $2$ и $4$ случает способов дозамостить полоску $A_{n-2}$, а в $3$ $A_{n-1}$ способов. Значит, $A_n=A_{n-1}+3A_{n-2}$. Характеристическое уравнение:

$\lambda^2-\lambda-3=0$

$\lambda=\frac{1\pm\sqrt{13}}{2}$

$A_n=C_1(\frac{1+\sqrt{13}}{2})^n+C_2(\frac{1-\sqrt{13}}{2})^n$

$A_1=1$, $A_2=4$.

$C_1\frac{1+\sqrt{13}}{2}+C_2\frac{1-\sqrt{13}}{2}=1$, $C_1(\frac{1+\sqrt{13}}{2})^2+C_2(\frac{1-\sqrt{13}}{2})^2=4$

$C_1=\frac{1}{2}+\frac{1}{2\sqrt{13}}$

$C_2=\frac{1}{2}-\frac{1}{2\sqrt{13}}$

$A_n=(\frac{1}{2}+\frac{1}{2\sqrt{13}})(\frac{1+\sqrt{13}}{2})^n+(\frac{1}{2}-\frac{1}{2\sqrt{13}})(\frac{1-\sqrt{13}}{2})^n=\frac{1}{2}((\frac{1+\sqrt{13}}{2})^n+(\frac{1-\sqrt{13}}{2})^n)+\frac{1}{2\sqrt{13}}((\frac{1+\sqrt{13}}{2})^n-(\frac{1-\sqrt{13}}{2})^n)$

$13^{\frac{31-1}{2}}=13^{15}=13\cdot 169^7=13\cdot 14\cdot 196^2=182\cdot 10^3=-40\cdot 100=-9\cdot 7 = -63 = -1$, то есть это квадратичный невычет.

$\sqrt{13}=x$, $\frac{1}{\sqrt{13}}=\frac{1}{x}=\frac{x}{13}$.

$2^{-1}=16\mod 31$, поскольку $2\cdot 16=32=1\mod 31$.

$A_n=16^{n+1}((1+x)^n+(1-x)^n)=16^{n+1}x((1+x)^n-(1-x)^n)$

$A_{30000}=16^{{30000}+1}((1+x)^{30000}+(1-x)^{30000})=16^{{30000}+1}x((1+x)^{30000}-(1-x)^{30000})=16\cdot 2^{120000}((1+x)^{30000}+(1-x)^{30000})+x((1+x)^{30000}-(1-x)^{30000})$

$2^{120000}=32^{24000}=1^{24000}=1\mod 31$

$(x\pm1)^{8}=(14\pm2x)^4=2^4(7\pm x)^4=2^4(62\pm14x)^2=2^4\cdot 14^2 x^2=2^5\cdot 2\cdot 49\cdot 13=98\cdot 13=5\cdot 13=65=3$

$(x\pm1)^{30000}=3^{3750}=27^{1250}=(-5)^{1250}=5^{1248}\cdot 5^2=125^{416}\cdot 25=1^{416}\cdot 25=25\mod 31$

$A_{30000}=16(25+25)+x(25-25)=16\cdot 50=25$


\section{Задача 4}
\hspace{5mm}
Да, так как это означала, что существует такой язык $V\in\mathcal{NPC}\cap co-\mathcal{NP}$. 

Тогда $\forall X\in co-\mathcal{NP}, \overline{X}\in \mathcal{NP}$, значит $\overline{X}$ может быть сведено к $V\in co-\mathcal{NP}$, т.~е. $\forall x\in \overline{X}\Leftrightarrow f(x)\in V$ или $\forall x\in X\Leftrightarrow f(x)\in \overline{V}$. Поскольку $V\in co-\mathcal{NP}$, то $\overline{V}\in \mathcal{NP}$. Получается каждый язык из $co-\mathcal{NP}$ сводится полиномиально к языку из $\mathcal{NP}$, значит, все языки из $co-\mathcal{NP}$ лежат в $\mathcal{NP}$.

С другой стороны тогда для каждого $Y\in \mathcal{NP}\rightarrow\overline{Y}\in co-\mathcal{NP}$ и $\overline{Y}\in \mathcal{NP}$, то есть $Y=\overline{\overline{Y}}\in co-\mathcal{NP}$. Следовательно классы совпадают.


\section{Задача 6}
\subsection{(i)}
\hspace{5mm}
То есть выпало $5$ орлов и $5$ решек, всего $P(5,5)=\frac{10!}{5!5!}$ случаев, чтобы это произошло, все случаев $2^{10}$. Ответ~---~$\frac{10!}{(5!)^2\cdot 2^{10}}=0,2461$

\subsection{(ii)}
\hspace{5mm}
То есть могло выпать либо $4$ решки, либо $3$, либо $2$, либо $1$, либо $0$. Значит всего подходящих случаев $|\Omega_{\phi}|=C_{10}^4+C_{10}^3+C_{10}^2+C_{10}^1+C_{10}^0=\frac{10\cdot9\cdot8\cdot7}{4!}+\frac{10\cdot9\cdot8}{3!}+\frac{10\cdot9}{2!}+\frac{10}{1!}+1$

$P(\phi)=\frac{|\Omega_{\phi}|}{|\Omega|}=0,3760$

\subsection{(iii)}
\hspace{5mm}
Значит, первые $5$ бросаний определяют всю серию бросаний, всего способов бросить $5$ раз $2^5$. Ответ~---~$\frac{2^5}{2^{10}}=\frac{1}{32}$.

\subsection{(iv)}
\hspace{5mm}



\section{Задача 7}
\subsection{(i)}
\hspace{5mm}
Тому что сумма выпавших костей оказалась $7$ способствуют элементарных $6$ событий: могли выпасть $(1,6),(2,5),(3,4),(4,3),(5,2),(6,1)$. И только в одном случае на первой выпало $6$. Значит, вероятность $\frac{1}{6}$.

\subsection{(ii)}
\hspace{5mm}
$\mathbb{E}(\max(X_1,X_2))+\mathbb{E}(\min(X_1,X_2))=\mathbb{E}(\min(X_1,X_2)+\max(X_1,X_2))=\mathbb{E}(X_1+X_2)=\mathbb{E}(X_1)+\mathbb{E}(X_2)=2\mathbb{E}(X)=2\cdot\frac{6+5+4+3+2+1}{6}=2\cdot\frac{21}{6}=\frac{21}{3}=7$.

\subsection{(iv)}
\hspace{5mm}
Проверим равенство $P(2k)\cdot P(3k)=P(6k)$: $\frac{1}{2}\cdot\frac{1}{3}=\frac{1}{6}$, значит, это действительно независимые события по определению.

\subsection{(v)}
\hspace{5mm}
Если есть $n$ вершин, то на них можно построить простой цикл $\frac{(n-1)!}{2}$ способами (произвольно выбранная вершина может вести в любую другую из $n-1$ вершин, ты в любую из оставшихся $n-2$ и т.~д. Остается только учесть, что каждый цикл был посчитан $2$ раза, когда его обходили в одну сторону и в другую). Всего графов на $n$ вершинах $2^{\frac{n^2-n}{2}}$, неориентированный граф взаимно однозначно задается симметричной относительно диагонали матрицей смежности, в которой на диагонали $0$, то есть необходимо заполнить $\frac{n^2-n}{2}$ клеток нулями и единицами. Значит вероятность равна $\frac{(n-1)!}{2^{\frac{n^2-n+2}{2}}}$.

$\lim\limits_{n\rightarrow\infty}\frac{(n-1)!}{2^{\frac{n^2-n+2}{2}}}
=\lim\limits_{n\rightarrow\infty}2^{\log\left(\frac{(n-1)!}{2^{\frac{n^2-n+2}{2}}}\right)}
\leq\lim\limits_{n\rightarrow\infty}2^{\log\left(\frac{(n-1)^{n-1}}{2^{\frac{n^2-n+2}{2}}}\right)}
=\lim\limits_{n\rightarrow\infty}2^{(n-1)log(n-1)-\frac{n^2-n+2}{2}}=\ =0 $


\section{Задача 8}
\hspace{5mm}
Пусть в урнах по $N$ шаров, в первой $n_1$ белых, а во второй $n_2$ белых шаров. Тогда вероятность, что все шары из первой урны будут белыми равна $(\frac{n_1}{N})^n$, вероятность, что все шары из второй урны будут белыми $(\frac{n_2}{N})^n$, а что все шары из второй урны будут черными $(\frac{N-n_2}{N})^n$. Значит, $(\frac{n_1}{N})^n=(\frac{n_2}{N})^n+(\frac{N-n_2}{N})^n,\ n>2$.

$n_1^n=n_2^n+(N-n_2)^n$, $n>2$. Но по великой теореме Ферма уравнения такого вида не имеют решений в целых чисел, при котором хотя бы одно из слагаемых не равно $0$. Значит, поскольку при $n_1=$ уравнение не имеет решений, то есть только $2$ варианта: $n_2=0$ или $n_2=N$, а $n_1=N$.

Итак, $n$ может быть любым, в первой урне обязательно только белые шары, а во второй либо все шары белые, либо все шары черные.



\section{Задача 9}
\hspace{5mm}
Построим бинарное дерево $S$ возможных событий. В узлах указывается какая сторона монеты выпала. Ветви, оканчивающиеся вершинам в двойных кружках не продолжаются, так как в них уже встретилось одно из слов. Вершинами $S$ обозначены поддеревья "подобные" всему дереву в том смысле, что если во всем дереве POP встречается раньше PPO с вероятностью $a$, то в этом поддереве POP встречается раньше с той же вероятностью. Действительно, в нижнем поддереве слова POP и PPO будут возникать в тех же местах, что и в целом дереве (это происходит из-за того, что ни одно из наших слов не начинается на O). То есть для того чтобы понять, что встретится с большей вероятностью нужно исследовать верхнюю ветвь. Дерево, чьё основание начинается в конце ветви POO, так же является "подобным", поскольку из-за предшествующих OO слова PPO и POP будет встречаться в тех же местах, что и во всем дереве. Ветви POP и PPO приводят к выпадению требуемых слов, поэтому остаются только верхняя ветвь, которая продолжается до бесконечности.
	
\begin{center}
	\begin{tikzpicture}[scale=0.2]
	\tikzstyle{every node}+=[inner sep=0pt]
	\draw [black] (5.2,-29.4) circle (3);
	\draw (5.2,-29.4) node {$S$};
	\draw [black] (16.4,-42.3) circle (3);
	\draw (16.4,-42.3) node {$O$};
	\draw [black] (14.1,-22.1) circle (3);
	\draw (14.1,-22.1) node {$P$};
	\draw [black] (23.2,-17) circle (3);
	\draw (23.2,-17) node {$P$};
	\draw [black] (23.2,-33) circle (3);
	\draw (23.2,-33) node {$O$};
	\draw [black] (28.9,-42.3) circle (3);
	\draw (28.9,-42.3) node {$S$};
	\draw [black] (36.3,-37) circle (3);
	\draw (36.3,-37) node {$O$};
	\draw [black] (36.3,-29.4) circle (3);
	\draw (36.3,-29.4) node {$P$};
	\draw [black] (36.3,-29.4) circle (2.4);
	\draw [black] (35.5,-21.4) circle (3);
	\draw (35.5,-21.4) node {$O$};
	\draw [black] (35.5,-21.4) circle (2.4);
	\draw [black] (36.3,-11.2) circle (3);
	\draw (36.3,-11.2) node {$P$};
	\draw [black] (47.7,-37) circle (3);
	\draw (47.7,-37) node {$S$};
	\draw [black] (48.3,-7.2) circle (3);
	\draw (48.3,-7.2) node {$P$};
	\draw [black] (48.3,-16.3) circle (3);
	\draw (48.3,-16.3) node {$O$};
	\draw [black] (48.3,-16.3) circle (2.4);
	\draw [black] (59.8,-3.7) circle (3);
	\draw (59.8,-3.7) node {$P$};
	\draw [black] (59.8,-12.2) circle (3);
	\draw (59.8,-12.2) node {$O$};
	\draw [black] (59.8,-12.2) circle (2.4);
	\draw [black] (71.4,-3.7) circle (3);
	\draw (71.4,-3.7) node {$...$};
	\draw [black] (7.52,-27.5) -- (11.78,-24);
	\fill [black] (11.78,-24) -- (10.84,-24.12) -- (11.48,-24.9);
	\draw [black] (7.17,-31.67) -- (14.43,-40.03);
	\fill [black] (14.43,-40.03) -- (14.29,-39.1) -- (13.53,-39.76);
	\draw [black] (16.72,-20.63) -- (20.58,-18.47);
	\fill [black] (20.58,-18.47) -- (19.64,-18.42) -- (20.13,-19.29);
	\draw [black] (16.02,-24.4) -- (21.28,-30.7);
	\fill [black] (21.28,-30.7) -- (21.15,-29.76) -- (20.38,-30.4);
	\draw [black] (19.4,-42.3) -- (25.9,-42.3);
	\fill [black] (25.9,-42.3) -- (25.1,-41.8) -- (25.1,-42.8);
	\draw [black] (25.94,-15.79) -- (33.56,-12.41);
	\fill [black] (33.56,-12.41) -- (32.62,-12.28) -- (33.03,-13.2);
	\draw [black] (26.02,-18.01) -- (32.68,-20.39);
	\fill [black] (32.68,-20.39) -- (32.09,-19.65) -- (31.75,-20.59);
	\draw [black] (26.09,-32.21) -- (33.41,-30.19);
	\fill [black] (33.41,-30.19) -- (32.5,-29.92) -- (32.77,-30.89);
	\draw [black] (26.07,-33.88) -- (33.43,-36.12);
	\fill [black] (33.43,-36.12) -- (32.81,-35.41) -- (32.52,-36.37);
	\draw [black] (39.3,-37) -- (44.7,-37);
	\fill [black] (44.7,-37) -- (43.9,-36.5) -- (43.9,-37.5);
	\draw [black] (39.15,-10.25) -- (45.45,-8.15);
	\fill [black] (45.45,-8.15) -- (44.54,-7.93) -- (44.85,-8.88);
	\draw [black] (39.06,-12.37) -- (45.54,-15.13);
	\fill [black] (45.54,-15.13) -- (45,-14.35) -- (44.61,-15.27);
	\draw [black] (51.05,-8.4) -- (57.05,-11);
	\fill [black] (57.05,-11) -- (56.51,-10.23) -- (56.12,-11.14);
	\draw [black] (51.17,-6.33) -- (56.93,-4.57);
	\fill [black] (56.93,-4.57) -- (56.02,-4.33) -- (56.31,-5.28);
	\draw [black] (62.8,-3.7) -- (68.4,-3.7);
	\fill [black] (68.4,-3.7) -- (67.6,-3.2) -- (67.6,-4.2);
	\end{tikzpicture}
\end{center}
	
Пусть вероятность, что при подбрасывании монетка первым встретится POP $a$, а то, что встретится PPO $b$. Тогда

$a=\frac{1}{2}a+(\frac{1}{8}a+\frac{1}{8})$, а $b=\frac{1}{2}b+(\frac{1}{8}b+\frac{1}{8}+\frac{1}{16}+\frac{1}{32}+\ldots)$

$a=\frac{1}{3}$

$3b=1+\frac{1}{2}+\frac{1}{4}+\cdot=2$, $b=\frac{2}{3}$

Ответ: PPO встретится раньше с большей вероятностью.






\section{Задача 10}
\subsection{(i)}
\hspace{5mm}
Построим новый генератор: будем генерировать пары чисел двумя последовательными случайными генерациями изначального генератора. Если сгенерировалось $00$, то генератор выдает $0$, если $10$ или $01$, то $1$. Если же $11$, то генерация пары чисел повторяется. Так продолжается, пока не выпадет одно из чисел $00$, $10$, $01$. Тогда вероятность сгенерировать $0$ будет $P(0)=\frac{1}{4}+\frac{1}{4}\cdot\frac{1}{4}+(\frac{1}{4})^2\cdot\frac{1}{4}+\ldots=\frac{1}{4}\frac{1}{1-\frac{1}{4}}=\frac{1}{4}\cdot\frac{4}{3}=\frac{1}{3}$, а $P(1)=\frac{2}{3}$ соответственно. В худшем случае генератор будет работать бесконечность, в среднем $E(t)=2\cdot\frac{3}{4}+4\cdot\frac{3}{4}\cdot\frac{1}{4}+6\cdot\frac{3}{4}\cdot(\frac{1}{4})^2+\ldots=\sum\limits_{k=1}^{\infty}2k\cdot3(\frac{1}{4})^k
=\frac{6}{4}\sum\limits_{k=1}^{\infty}k(\frac{1}{4})^{k-1}
=\frac{3}{2}\sum\limits_{k=0}^{\infty}\left(x^k\right)'|_{x=\frac{1}{4}}
=\frac{3}{2}\left(\sum\limits_{k=0}^{\infty}x^k\right)'|_{x=\frac{1}{4}}
=\frac{3}{2}\cdot\left(\frac{1}{1-x}\right)'|_{x=\frac{1}{4}}=
\frac{3}{2}\frac{1}{(1-x)^2}|_{x=\frac{1}{4}}=\frac{3\cdot16}{2\cdot 9}=\frac{8}{3}$. В среднем одна генерация этого генератора будет работать $2\frac{2}{3}$ от одной генерации исходного генератора.

\subsection{(ii)}
\hspace{5mm}
Будем так же генерировать пары битов исходным генератором. Если сгенерировалось $11$ (с вероятностью $\frac{4}{9}$), то генератор выдает $1$. Если сгенерировалось $10$ или $01$ (с вероятностью $\frac{2}{9}+\frac{2}{9}=\frac{4}{9}$), то генератор выдает $0$. Поэтому генератор будет генерировать $0$ и $1$ с равной вероятностью. Если сгенерировалось $00$, то генерация пары битов повторяется, и так пока не выпадет $11$, $01$ или $10$.










\end{document} % конец документа