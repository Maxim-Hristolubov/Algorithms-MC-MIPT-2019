\documentclass[a4paper,12pt]{article} % добавить leqno в [] для нумерации слева

%%% Работа с русским языком
\usepackage{cmap}					% поиск в PDF
\usepackage{mathtext} 				% русские буквы в формулах
\usepackage[T2A]{fontenc}			% кодировка
\usepackage[utf8]{inputenc}			% кодировка исходного текста
\usepackage[english,russian]{babel}	% локализация и переносы

%%% Дополнительная работа с математикой
\usepackage{amsmath,amsfonts,amssymb,amsthm,mathtools} % AMS
\usepackage{icomma} % "Умная" запятая: $0,2$ --- число, $0, 2$ --- перечисление

%% Номера формул
\mathtoolsset{showonlyrefs=true} % Показывать номера только у тех формул, на которые есть \eqref{} в тексте.

%% Раисунки (автоматы)
\usepackage{tikz}

%% Шрифты
\usepackage{euscript}	 % Шрифт Евклид
\usepackage{mathrsfs} % Красивый матшрифт

%% Свои команды
\DeclareMathOperator{\sgn}{\mathop{sgn}}
\usepackage{ dsfont }

%% Перенос знаков в формулах (по Львовскому)
\newcommand*{\hm}[1]{#1\nobreak\discretionary{}
	{\hbox{$\mathsurround=0pt #1$}}{}}

%%% Заголовок
\author{771 группа, Христолюбов Максим}
\title{Домашнее задание по АМВ}
\date{\today}

\begin{document} % конец преамбулы, начало документа
	
	\maketitle

*Я только в конце заметил, что $\psi$~---~это конкретная функция из файлика а не абстрактная функция, я в подводящих задачах $(i) (ii)$ везде использовал произвольную придуманную функцию.

\section{Задача 1}
\hspace{5mm}
Любой дезъюнкт, в котором менее 3 литералов можно свести к конъюнкции дезъюнктов из ровно 3 литералов:

$x\vee y=(x\vee y\vee z)\wedge (x\vee y\vee\overline z)$

$x=(x\vee y\vee z)\wedge (x\vee y\vee\overline z)\wedge (x\vee\overline y\vee z)\wedge (x\vee\overline y\vee\overline z)$

Значит, любой $x\in$3-SAT можно свести к $f(x)\in$РОВНО-3-ВЫПОЛНИМОСТЬ, причем длина каждого дезъюнкта увеличиться не более, чем в константное кол-во раз, следовательно $f(|x|)\leq c|x|$ и сводимость полиномиальна.

\section{Задача 2}
\subsection{34 (i)}
\hspace{5mm}
Пусть $A_{\psi}=\{\{x_1,\overline x_1\}, \{x_2,\overline x_2\},\{x_3,\overline x_3\},\{x_1,x_2,\overline x_3\},\{\overline x_1,x_2,x_3\}\}$. Тогда его протыкающее множество будет $\{x_1,x_2,x_3\}$.

\subsection{34 (ii)}
\hspace{5mm}
Покажем, что мощность любого протыкающего множества системы $A_{\psi}$, построенного соответствующим образом, не превышает $n$. Это действительно так, ведь в множествах системы есть $n$ множеств вида $\{x_i,\overline x_i\}$. В протыкающее множество должен входить как минимум один элемент из каждого из них, причем в них нет общих переменных. Значит, в протыкающее множество обязательно входит $n$ разных элементов, каждый из которых взят из своего $\{x_i,\overline x_i\}$.

\subsection{Сведение}
\hspace{5mm}
По формуле из $SAT$, предварительно переведя ее в КНФ, можно построить по соответствующей схеме систему множеств $A_{\psi}$. Покажем, что $A_{\psi}$ имеет протыкающее множество мощности $n$ тогда и только тогда, когда формула выполнима.
 
 Если формула выполнима, тогда существует такой набор, который обращает ее в $1$. На его основе можно построить протыкающее множество: те переменные, которые входят в набор как $1$ включим в протыкающее множество, а те, что входят как $0$, включим с отрицанием $\overline x_i$. Полученное множество, очевидно, протыкает все $\{x_i,\overline x_i\}$. Все переменные, входящие в это множество по построению обращаются на наборе в $1$, а так как в каждом дезъюнкте есть хотя бы $1$ переменная, равная $1$, в силу выполнимости формулы на этом наборе, то во все множества системы (которые соответствуют дезъюнктам) входит хотя бы $1$ переменная из полученного множество, значит, оно~---~протыкающее множество.

Если существует протыкающее множество мощности $n$, то в него входит или $x_i$, или $\overline x_i$ для каждого $i$. Рассмотрим набор, на котором все элементы протыкающего множества равны $1$ ($x_i=1$, если в нем $x_i$, и $x_i=0$, если в нем $\overline x_i$). Тогда в каждое множество системы будет входить как минимум одна переменная равная $1$, а значит, все дезъюнкты будут в себя включать хотя бы одну $1$ и сами будут равны $1$. Значит, и вся КНФ будет равна $1$, то есть существует набор обращающий ее в $1$, и формула выполнима.




\section{Задача 3}
\subsection{35 (i)}
\hspace{5mm}
В приведенном нижу графе вершины в двойных кружках~---~литеральные, в одинарных кружках~---~дезъюнктивные, а закрашенные в синий вершины~---~вершины, входящие в $(n+2m)$-вершинное покрытие, $n$~---~кол-во литералов, $m$~---~кол-во дезъюнктов.

\begin{center}
	\begin{tikzpicture}[scale=0.2]
	\tikzstyle{every node}+=[inner sep=0pt]
	\draw [blue] (35.3,-26.5) circle (3);
	\draw (35.3,-26.5) node {$x_1$};
	\draw [blue] (35.3,-26.5) circle (2.4);
	\draw [black] (46.8,-26.5) circle (3);
	\draw (46.8,-26.5) node {$\overline x_1$};
	\draw [black] (46.8,-26.5) circle (2.4);
	\draw [blue] (39.8,-36.1) circle (3);
	\draw (39.8,-36.1) node {$x_2$};
	\draw [blue] (39.8,-36.1) circle (2.4);
	\draw [black] (31.1,-36.1) circle (3);
	\draw (31.1,-36.1) node {$\overline x_2$};
	\draw [black] (31.1,-36.1) circle (2.4);
	\draw [blue] (23.1,-19.7) circle (3);
	\draw (23.1,-19.7) node {$x_3$};
	\draw [blue] (23.1,-19.7) circle (2.4);
	\draw [black] (54.4,-19.7) circle (3);
	\draw (54.4,-19.7) node {$\overline x_3$};
	\draw [black] (54.4,-19.7) circle (2.4);
	\draw [black] (51.3,-39.5) circle (3);
	\draw (51.3,-39.5) node {$x_1$};
	\draw [black] (19.3,-39.5) circle (3);
	\draw (19.3,-39.5) node {$x_1$};
	\draw [blue] (29.5,-45.3) circle (3);
	\draw (29.5,-45.3) node {$\overline x_2$};
	\draw [blue] (41.3,-45.3) circle (3);
	\draw (41.3,-45.3) node {$x_2$};
	\draw [blue] (61.8,-45.3) circle (3);
	\draw (61.8,-45.3) node {$\overline x_3$};
	\draw [blue] (10.1,-45.3) circle (3);
	\draw (10.1,-45.3) node {$x_3$};
	\draw [black] (59.17,-43.85) -- (53.93,-40.95);
	\draw [black] (53.93,-40.95) -- (59.17,-43.85);
	\draw [black] (48.7,-41.01) -- (43.9,-43.79);
	\draw [black] (43.9,-43.79) -- (48.7,-41.01);
	\draw [black] (44.3,-45.3) -- (58.8,-45.3);
	\draw [black] (58.8,-45.3) -- (44.3,-45.3);
	\draw [black] (26.5,-45.3) -- (13.1,-45.3);
	\draw [black] (13.1,-45.3) -- (26.5,-45.3);
	\draw [black] (21.91,-40.98) -- (26.89,-43.82);
	\draw [black] (26.89,-43.82) -- (21.91,-40.98);
	\draw [black] (16.76,-41.1) -- (12.64,-43.7);
	\draw [black] (12.64,-43.7) -- (16.76,-41.1);
	\draw [black] (38.3,-26.5) -- (43.8,-26.5);
	\draw [black] (51.4,-19.7) -- (26.1,-19.7);
	\draw [black] (34.1,-36.1) -- (36.8,-36.1);
	\draw [black] (36.8,-36.1) -- (34.1,-36.1);
	\draw [black] (26.1,-19.7) -- (51.4,-19.7);
	\draw [black] (43.8,-26.5) -- (38.3,-26.5);
	\draw [black] (11.46,-42.63) -- (21.74,-22.37);
	\draw [black] (21.74,-22.37) -- (11.46,-42.63);
	\draw [black] (21.63,-37.61) -- (32.97,-28.39);
	\draw [black] (32.97,-28.39) -- (21.63,-37.61);
	\draw [black] (37.63,-28.39) -- (48.97,-37.61);
	\draw [black] (48.97,-37.61) -- (37.63,-28.39);
	\draw [black] (55.23,-22.58) -- (60.97,-42.42);
	\draw [black] (60.97,-42.42) -- (55.23,-22.58);
	\draw [black] (40.82,-42.34) -- (40.28,-39.06);
	\draw [black] (40.28,-39.06) -- (40.82,-42.34);
	\draw [black] (30.59,-39.06) -- (30.01,-42.34);
	\draw [black] (30.01,-42.34) -- (30.59,-39.06);
	\end{tikzpicture}
\end{center}

\subsection{35 (ii)}
\hspace{5mm}
Мощность любого покрытия не менее $n+2m$, так как в графе есть $n$ ребер, соединяющие литералы и их отрицания, их всех нужно покрыть, а значит, в вершинное подпокрытие обязательно входит литерал или его отрицание, это $n$ вершин. Так же обязательно нужно покрыть ребра, соединяющие дезъюнктные вершины, поэтому в вершинное покрытие должны входить минимум $2$ из каждой тройки вершин, образованной каждым дезъюнктом, а это $2m$ вершин. Значит, мощность вершинного покрытия не меньше $n+2m$.

\subsection{Сведение}
\hspace{5mm}
Покажем, что если КНФ выполнима, то найдется вершинное покрытие мощностью $n+2m$ в графе, построенным как указано в задаче 35. Так как КНФ выполнима, то в каждый дезъюнкт входит как минимум $1$ переменная, обращающаяся в $1$ на этои наборе. В таком случае можно выбрать покрытие так: те литеральные вершины, которые имеют значение $1$ на наборе, обращающем КНФ в тождество, нужно включить в покрытие. Так же нужно включить по $2$ дезъюнктивные вершины из каждого дезъюнкта. Поскольку в каждом дезъюнкте есть вершина, обращающаяся в $1$, то она соединена ребром с литеральной вершиной, уже входящей в покрытие, а значит, ребро между ними уже покрыто. Поэтому включив в покрытие $2$ остальные вершины дезъюнкта, можно добиться покрытия всех ребер между литеральными и дезъюнктивными вершинами. Ребра только между литеральными и только между дезъюнктивными уже покрыты. Значит, такое $(n+2m)$-вершинное покрытие в таком графе существует.

Ранее было показано, что размеры любого вершинного покрытия не меньше $n+2m$, а значит, если нашлось такое покрытие, то оно включает по $2$ вершины из каждого дезъюнкта и по $1$ из каждой пары литеральных вершины. Можно составить набор, обращающий КНФ в $1$: если в паре литеральных вершин выбрана $x_i$ в покрытие, то $x_i=1$, иначе $x_i=0$. На этом наборе в каждом дезъюнкте найдется множитель, равный $1$, так как в каждую тройку дезъюнктных вершин входит вершин, соединенная с выбранной в покрытие литеральной вершиной, которая будет обращаться в $1$. Значит, КНФ выполнима.

Сведение доказано, так как КНФ выполнима тогда и только тогда, когда соответствующий граф имеет вершинное покрытие размером $n+2m$.

\section{Задача 4}
\subsection{36 (i)}
\hspace{5mm}
Пример построенного по КНФ графа с выделенным синим кликой.
	\begin{center}
		\begin{tikzpicture}[scale=0.2]
		\tikzstyle{every node}+=[inner sep=0pt]
		\draw [black] (57.2,-30.4) circle (3);
		\draw (57.2,-30.4) node {$\overline x_1$};
		\draw [blue] (57.2,-20.5) circle (3);
		\draw (57.2,-20.5) node {$\overline x_2$};
		\draw [black] (57.2,-10.3) circle (3);
		\draw (57.2,-10.3) node {$x_3$};
		\draw [black] (19.2,-30.4) circle (3);
		\draw (19.2,-30.4) node {$x_1$};
		\draw [black] (19.2,-20.5) circle (3);
		\draw (19.2,-20.5) node {$ x_2$};
		\draw [blue] (19.2,-10.3) circle (3);
		\draw (19.2,-10.3) node {$x_3$};
		\draw [blue] (22.1,-11.08) -- (54.3,-19.72);
		\draw [black] (21.85,-11.7) -- (54.55,-29);
		\draw [black] (22.2,-10.3) -- (54.2,-10.3);
		\draw [black] (22.1,-19.72) -- (54.3,-11.08);
		\draw [black] (22.1,-21.26) -- (54.3,-29.64);
		\draw [black] (21.9,-29) -- (54.56,-11.73);
		\draw [black] (22.1,-30) -- (54.3,-21.6);
		\end{tikzpicture}
	\end{center}
	
\subsection{36 (ii)}
\hspace{5mm}
В клике не могут вершины из одного дезъюнкта по построению графа, всего дезъюнктов $m$, значит, мощность клики не больше $m$.

\subsection{Сведение}
\hspace{5mm}
Если в графе есть клика мощностью $m$, то она проходит по $1$ разу через каждую тройку, соответствующие дезъюнктам, причем в ней не будет вершин, являющиеся отрицанием друг друга. Значит, по этой клике можно составить набор, обращающий КНФ в $1$. Для каждой вершины, входящей в клику, переменная в ней будет обращаться в $1$, тогда каждый дезъюнкт будет равен $1$, и КНФ будет выполнимо.

Если КНФ выполнимо, то включим в клику по $1$ вершине из каждого дезъюнкта, так чтобы они обращались в $1$ на наборе, на котором КНФ выполнимо. Все эти вершины будут действительно попарно соединены, так как в них не будет отрицаний друг друга,  а так же они будут из разных дезъюнктов.

\section{Задача 5}
\subsection{37 (i) и (ii)}
\hspace{5mm}
$\psi=p\vee q\vee r$, а $\overline\psi=p\wedge q\wedge r\wedge d\wedge (p\vee \overline q)\wedge(\overline p\vee\overline r)\wedge(\overline q\vee\overline r)\wedge(\overline p\vee \overline d)\wedge(q\vee \overline d)\wedge(r\vee \overline d)$, если $p=q=r=0$, тогда в нем выполнено не более $6$ дезъюнктов (при $d=0$ ровно $6$), а в остальных случаях существует $d$, что выполнено $7$ дезъюнктов, причем никогда не выполнено больше $7$, например при $p=q=r=1,d=0$ выполнено $7$ дезъюнктов.


\subsection{Сведение}
\hspace{5mm}
Дезъюнкт $\psi$ обращается в $1$ только тогда, когда существует $d$, что в $\overline\psi$ выполнено $7$ дезъюнктов, значит, это можно использовать для проверки выполнимости $\psi$. Для любого дезъюнкта КНФ можно составить соответствующую конъюнкцию и образовать из них новую КНФ размером $7n,n$~---~кол-во дезъюнкций в исходной КНФ. 

Тогда если исходная КНФ выполнима, то найдется $7n$ выполненных в новой КНФ дезъюнкций. 

Если найдутся $7n$ выполнимых дезъюнкций в новой КНФ, то и исходная КНФ выполнима, так как это возможно только, когда в каждой десятке дезъюнкций (каждая из которых образована дезъюнктом) выполнено 7 дезъюнкций, а значит, существуте набор, на котором каждый дезъюнкт исходной КНФ выполнен, и КНФ выполнима.





\section{Задача 6}
\hspace{5mm}
Пусть есть полиномиальный алгоритм, проверяющий можно ли раскрасить граф в $3$ цвета за $t(x)$. Построим алгоритм, находящий эту раскраску за полиномиальное время. Рассмотрим пару вершин $i$ и $j$. Модифицируем $x$ в $x(ij)$, добавив к нему пару вершин $a,b$ и ребра $ab,ia,ib,ja,jb$. При раскрашивании нового графа $a$ и $b$ должны иметь разные цвета, а так же цвета $i$ и $j$ должны быть отличными от их цвета, а так как всего $3$ цвета, то $i$ и $j$  будут одного цвета. То же самое можно сделать для $3$ и более вершин, например для $i,j,k$, добавив вершины $a,b$ и ребра $ab,ia,ja,ka,ib,jb,kb$. Следовательно проверкой такого модифицированного графа на принадлежность $3$-COLOR можно за полиномиальное время определить существует ли такая раскраска графа $x$, что некоторое множество вершин одного цвета. 

Проверим, что граф $x$ действительно можно раскрасить. В алгоритме без ограничения общности закрасим вершину $1$ в $0$-цвет и будем проверять $x(1i),i\in N$, пока не найдем вершину $p_1$, которую тоже можно закрасить в $0$. После этого будем проверять $x(1 p_1 i), i>p_1$, пока не найдем еще одну вершину $p_2$ цвета $0$. Тогда проверяем $x(1 p_1 p_2 i), i>p_2$ и так, пока в графе не останется вершин, которые можно закрасить в $0$. После этого закрасим не закрашенную вершину $v$ в $1$-цвет и будем проверять $x(v i)$ аналогично цвету $0$, ища какие еще вершины можно закрасить в $1$. После одну из оставшихся вершин нужно закрасить в $2$-цвет и повторить операцию. В итоге, все вершины будут закрашены, так как больше ни в какие цвета граф вершины $x$ раскрасить нельзя, а еще в начале было установленно, что граф раскрасить можно в $3$ цвета. Всего было не более $3n$ ($n$~---~число вершин графа) вызовов проверок на раскрашиваемость, размер модифицированного графа увеличивался не более чем в $2$ раза, значит, приведенный алгоритм работает за полиномиальное время.

\section{Задача 7}
\hspace{5mm}
Формула имеет вид конъюнкции

$(y_0\equiv f_0)(y_1\equiv f_1)\ldots(y_n\equiv f_n)y_n, f_k$~---~формулы от $y_i, x_i$. Каждому узлу схемы соответствует собственная скобка. При вычислении значении схемы при определенном наборе высчитывается булевое значение узлов, основываясь на булевых значениях их потомков ($x_i$~---~листья). Будем делать то же самое в формуле, подставив в него этот набор. Значение $f_k$ соответствует тому значению, которое принимает узел. Если бы $y_k$ имело бы другое значение, то фвся конъюнкция оказалась бы заведомо $0$, поэтому $y_k$ принимает то же значение, что и соответствующий узел. Когда $y_k$ войдет в формулу $f_p$, это будет значить, что какой-то узел вычисляет свое значение, основываясь на значении узла, соответствующем $f_k$. В итоге вычислении значений узлов с итоговым вычислением значения всей схемы эквивалентно определению значения набора $y_1\ldots y_n$, такого $y_n$ равно значению схемы. Так как на таком наборе все множители, кроме $y_n$ равны $1$, то только значение $y_n$ определяет значение выражения, то есть значение выражение всегда равно значению схемы. Иначе говоря, если схема выполнима, то существует набор $\overline x$ и набор $\overline y$, обращающий формулу в $1$. Если же схема не выполнима, то есть на всех наборах принимает значение $0$, то даже если подобрать такой набор $\overline y$, чтобы все множители были равны $1$, то обязательно $y_n=0$, следовательно, формула не выполнима.




\end{document} % конец документа